% Donald Knuth's Turingol attributed grammar.
% LaTeX'd by Peter Gammie, peteg@cse.unsw.edu.au
% October, 1998.

\documentclass[a4paper]{article}
\usepackage{a4wide}

\begin{document}

\title{excerpt from \\ \bfseries{Semantics of Context-Free Languages}}
\author{\textsc{Donald E. Knuth}\\
	California Institute of Technology}
\date{1968}
\maketitle

\begin{abstract}
``Meaning'' may be assigned to a string in a context-free language by defining
``attributes'' of the symbols in a derivation tree for that string. The
attributes can be defined by functions associated with each production in the
grammar. This paper examines the implications of this process when some of the
attributes are ``synthesized'', i.e., defined solely in terms of attributes of
the \emph{descendants} of the corresponding nonterminal symbol, while other
attributes are ``inherited'', i.e., defined in terms of attributes of the
ancestors of the nonterminal symbol. An algorithm is given which detects when
such semantic rules could possibly lead to circular definition of some
attributes. An example is given of a simple programming language defined with
both inherited and synthesized attributes, and the method of definition is
compared to other techniques for formal specification of semantics which have
appeared in the literature.
\end{abstract}

Only section 4 is included, as the rest of the document is peripheral to
Turingol. These excerpts are from the article ``Semantics of context-free
languages'' \emph{Mathematical Systems Theory} (1968), 127--145. 
Errata, \emph{Mathematical Systems Theory} (1971), 95--96.

\setcounter{section}{3}

\section{A simple programming language.}

Now let us consider an example of how the above techniques of semantic
definition can be applied to programming languages. For simplicity let us study
a formal definition of a little language that describes Turing machine
programs.

A Turing machine (in the classical sense) processes an infinite tape which may
be thought of as divided into squares; the machine can read or write characters
from a finite alphabet on the tape in the square which is currently being
scanned, and it can move the scanning position to the left or right. The
following program, for example, adds unity to an integer number, assuming that
the square just to the right of the number is to be scanned at the beginning
and end of the program:

\begin{center}
\begin{minipage}{13cm}
{\bfseries tape alphabet is} $blank, one, zero, point$; \\
{\bfseries print} ``$point$''; \\
{\bfseries go to} $carry$; \\
$test:$ {\bfseries if the tape symbol is} ``$one$'' {\bfseries then} \\
\hspace*{5em}\{{\bfseries print} ``$zero$''; $carry:$ {\bfseries move left one square; go to} $test$\}; \\
{\bfseries print} ``$one$''; \\
$realign:$ {\bfseries move right one square;} \\
\hspace*{2.5em}{\bfseries if the tape symbol is} ``$zero$'' {\bfseries then go to} $realign$.
\end{minipage}
\end{center}
(It is hoped that the reader will find this programming language sufficiently
self-explanatory that he understands it before any formal definition of the
language is given, although of course this is not necessary. The above program
is not intended as an example of good programming, rather as an example of the
features of the simple language considered in this section.)

Since every programming language must have a name, let us call the language
Turingol. Any well-formed Turingol program defines a program for a Turing
machine; let us say a Turing machine program consists of

\begin{center}
\begin{minipage}{5cm}
a set $Q$ of ``states'';\\
a set $\Sigma$ of ``symbols'';\\
an ``initial state'' $q_0 \in Q$;\\
a ``final state'' $q_\infty \in Q$;
\end{minipage}
\end{center}
and a ``transition function'' $\delta$ which maps $(Q - \{q_\infty\}) \times
\Sigma$ into $\Sigma \times \{-1,0,+1\} \times Q$. If $\delta(q,s) = (s',k,q')$
we may say informally that, if the machine is in state $q$ and scanning symbol
$s$, it will print symbol $s'$, move $k$ spaces to the right(meaning one space
to the left if $k=-1$), and go into state $q'$. Move formally, a Turing machine
program defines a computation on any ``initial tape contents'', i.e., on any
doubly infinite sequence

\begin{equation}
	\cdots, a_{-3}, a_{-2}, a_{-1}, a_0, a_1, a_2, a_3, \cdots
\end{equation}
of elements of $\Sigma$, as follows: At any moment of the computation there is a
``current state'' $q \in Q$ and an integer-valued ``tape position'' $p$;
initially $q=q_0$ and $p=0$. If $q \neq q_\infty$, and if $\delta(q,
a_p)=(s',k,q')$, the computation proceeds by replacing the value of $a_p$ by
$s'$, then by replacing $p$ by $p+k$ and $q$ by $q'$. If $q=q_\infty$, the
computation terminates. (The computation might not terminate; for program (4.1)
this happens if and only if $a_j =$ ``$one$'' for all $j<0$.)

Now we have a precise definition of Turing machine programs, we wish to define
the Turing machine program corresponding to any given Turingol program (and at
the same time to define the syntax of Turingol). For this purpose it is
convenient to introduce a few abbreviation conventions.

\begin{enumerate}

\item The semantic rule {\bfseries ``include $x$ in $B$''} associated with a
production will mean that $x$ is to be a member of set $B$, where $B$ is an
attribute of the start symbol $S$ of the grammar. The value of $B$ will be the
set of all $x$ for which such a semantic rule has appeared corresponding to each
appearance of the production in the derivation tree. (The rule may be regarded
as an abbreviation for the semantic rule

\begin{equation}
	B(X_{p0}) = \bigcup_{j=1}^{n_p} B(X_{pj}) \cup
		\{x| \mbox{{\bfseries ``include }} x {\mbox {\bfseries\ in }}
		B\mbox{{\bfseries ''} is associated with the } p \mbox{-th production}\}
\end{equation}
added to each production, with a set $B$ added as a synthesized attribute of
each nonterminal symbol, and $B(x)$ the empty set for each terminal symbol.
These rules clearly make $B(S)$ the desired set.)

\item The semantic rule {\bfseries ``define $f(x) = y$''} associated with a
production will mean that $y$ is to be the value of the function $f$ evaluated
at $x$, where $f$ is an attribute of the start symbol $S$ of the grammar. If
two rules occur defining $f(x)$ for the \emph{same} value of $x$, this is an
error condition, and any derivation tree which allows this condition to occur
may be said to be \emph{malformed}. Furthermore, $f$ may be used as a function
in any other semantic rules, with the proviso that $f(x)$ may only appear when
$f$ has been defined at $x$; any derivation tree which calls for an undefined
value of $f(x)$ is \emph{malformed}. (This type of rule is important, for
example, to ensure that there is agreement between the declaration and use of
identifiers. In the example below this convention implies that programs are
malformed if the same identifier is used twice as a label or if a {\bfseries go
to} statement specifies an identifier which is not a statement label. The rule
may essentially be thought of as {\bfseries ``include $(x, y)$ in $f$''}, as in
1, if $f$ is regarded as a set of ordered pairs; additional checks for
malformedness are also included. We may regard ``well-formed or malformed'' as
an attribute of $S$; appropriate semantic rules analogous to (4.3) which
completely specify this {\bfseries ``define $f(x) = y$''} convention are
readily constructed and left to the reader.)

\item The function {\bfseries ``newsymbol''} appearing in any semantic rule
will have, as its value, an abstract element which for each evaluation of
{\bfseries ``newsymbol''} is different from the abstract element produced by
other evaluations of {\bfseries newsymbol}. (This convention can readily be
expressed in terms of other semantic rules, e.g., by making use of the $l$
attributes of (2.3) which has a different value at each node of a tree. The
function {\bfseries newsymbol} serves as a convenient source of ``raw
material'' for constructing sets.)

\end{enumerate}

We have observed that conventions 1, 2, 3 can be replaced by other constructions
of semantic rules which do not use these conventions, so they are not
``primitives'' for semantics. But they are of fairly wide utility, since they
correspond to concepts which are often needed, so they may be regarded as
fundamental aspects of the techniques for semantic definition presented in this
paper. The effect of using there conventions is to reduce the number of
attributes that explicitly mentioned and to avoid unnecessarily long rules.

Now it is a simple matter to present a formal definition of the syntax and
semantics of Turingol.

Nonterminal symbols: $P$ (program), $S$ (statement), $L$ (list of statements),
	$I$ (identifier), $O$ (orientation), $A$ (alphabetic character), $D$
	(declaration).

Terminal symbols: $a\ b\ c\ d\ e\ f\ g\ h\ i\ j\ k\ l\ m\ n\ o\ p\ q\ r\ s\ t\
	u\ v\ w\ x\ y\ z$\ .\ ,\ :\ ;\ ``\ ''\ \{\ \}\ {\bfseries tape alphabet is
	print go to if the symbol then move left right one square}

Start symbol: $p$

Attributes:

\begin{tabular}{cp{5cm}p{5cm}}
\multicolumn{1}{c}{\emph{Name of attribute}} &
\multicolumn{1}{c}{\emph{Type of value}} &
\multicolumn{1}{c}{\emph{Purpose}} \\
$Q$ & Set & States of the program \\
$\Sigma$ & Set & Symbols of the program \\
$q_0$ & Element of $Q$ & Initial state \\
$q_\infty$ & Element of $Q$ & Final state \\
$\delta$ & Function from $(Q - q_\infty) \times \Sigma$ into $\Sigma \times
	\{-1,0,+1\} \times Q$ & Transition function \\
label & Function from strings of letters into elements of $Q$ & State table for
	statement labels \\
symbol & Function from strings of letters into elements of $S$ & Symbol table to
	tape symbols \\
follow & Element of $Q$ & State immediately following statement or list
	of statements \\
d & $\pm 1$ & Direction \\
text & String of letters & Identifier \\
start & Element of $Q$ & State at the beginning of a statement or list of
	statements (an inherited attribute).
\end{tabular}

Productions and semantics: See Table 1.

Notice that two states correspond to each statement $S$: start($S$) is the
state corresponding to the first instruction of the statement (if any), and it
is an inherited attribute of $S$; follow($S$) is the state which ``follows''
the statement, the state which is normally reached after the statement is
executed. In the case of a ``goto statement'', however, the program does not
transfer to follow($S$), since the action of the statement is to change control
to another place; follow($S$) may be said to follow statement $S$
``statically'' of ``textually'', not ``dynamically'' during a run of the
program.

In Table 1, follow($S$) is a synthesized attribute; it is possible to give
similar semantic rules in which follow($S$) is inherited, although a less
efficient program would be obtained for null statements (see Rule 4.4).
Similarly, both start($S$) and follow($S$) could be synthesized attributes, but
at the expense of additional instructions in the Turing machine program for
statement lists (Rule 6.2).

This example would be somewhat simpler if we had used a less standard
definition of Turing machine instructions. The definition we have used requires
reading, printing, and shifting in each instruction, and also makes the Turing
machine into a kind of ``one-plus-one-address computer'' in which each
instruction specifies the location (state) of the next instruction.

\newpage

\newcommand{\syntax}[1]{\parbox{2.15cm}{\raggedright {\sloppy #1 }}}
\newcommand{\example}[1]{\parbox{3.7cm}{\raggedright {\sloppy #1 }}}
\newcommand{\semantics}[1]{\parbox{5cm}{\raggedright {\sloppy #1 }}}

\begin{center}
{\bfseries Table 1.}

\begin{tabular}{llp{2.15cm}p{3.7cm}p{5cm}}
\hline\hline
\multicolumn{1}{c}{\emph{Description}} &
\multicolumn{1}{c}{\emph{No.}} &
\multicolumn{1}{c}{\emph{Syntactic Rule}} &
\multicolumn{1}{c}{\emph{Example}} &
\multicolumn{1}{c}{\emph{Semantic Rules}}\\
\hline
Letters &
 1.1 &
 \syntax{$A \rightarrow a$} &
 \example{$a$} &
 \semantics{text($A$) = $a$.} \\
 & \multicolumn{2}{c}{\dots} & (similarly for all letters) \\
&
 1.26 &
 \syntax{$A \rightarrow z$} &
 \example{$z$} &
 \semantics{text($A$) = $z$.} \\
Identifiers &
 2.1 &
 \syntax{$I \rightarrow A$} &
 \example{$m$} &
 \semantics{text($I$) = text($A$).} \\
&
 2.2 &
 \syntax{$I \rightarrow IA$} &
 \example{$marilyn$} &
 \semantics{text($I$) = text($I$)text($A$).} \\
\hline
Declarations &
 3.1 &
 \syntax{$D \rightarrow$ {\bfseries tape alphabet is} $I$} &
 \example{{\bfseries tape alphabet is} $marilyn$} &
 \semantics{{\bfseries define} symbol(text($I$)) = {\bfseries newsymbol;} \\
	{\bfseries include} symbol(text($I$)) {\bfseries in} S.} \\
&
 3.2 &
 \syntax{$D \rightarrow D, I$} &
 \example{{\bfseries tape alphabet is} $marilyn, jayne, birgitta$} &
 \semantics{{\bfseries define} symbol(text($I$)) = {\bfseries newsymbol;} \\
	{\bfseries include} symbol(text($I$)) {\bfseries in} S.} \\
\hline
Print statement &
 4.1 &
 \syntax{$S \rightarrow$ {\bfseries print} ``$I$''} &
 \example{{\bfseries print} ``$jayne$''} &
 \semantics{{\bfseries define} $\delta$(start($S$), $s$) = (symbol(text($I$)), 0, follow($S$)) for all $s \in \Sigma$; \\
	follow($S$) = {\bfseries newsymbol;} \\
	{\bfseries include} follow($S$) in $Q$.} \\
\hline
Move statement &
 4.2 &
 \syntax{$S \rightarrow$ {\bfseries move $O$ one square}} &
 \example{{\bfseries move left one square}} &
 \semantics{{\bfseries define} $\delta$(start($S$),$s$)=($s$, d($O$), follow($S$)) for all $s \in \Sigma$; \\
	follow($S$) = {\bfseries newsymbol;} \\
	{\bfseries include} follow($S$) {\bfseries in} $Q$.} \\
&
 4.2.1 &
 \syntax{$O \rightarrow$ {\bfseries left}} &
 \example{{\bfseries left}} &
 \semantics{d($O$) = $-1$.} \\
&
 4.2.2 &
 \syntax{$O \rightarrow$ {\bfseries right}} &
 \example{{\bfseries right}} &
 \semantics{d($O$) = $+1$.} \\
\hline
Go statement &
 4.3 &
 \syntax{$S \rightarrow$ {\bfseries go to} $I$} &
 \example{{\bfseries go to} $boston$} &
 \semantics{{\bfseries define} $\delta$(start($S$,$s$) = ($s$, 0, label(text($I$)) for all $S \in \Sigma$; \\
	follow($S$) = {\bfseries newsymbol;} \\
	{\bfseries include} follow($S$) {\bfseries in} $Q$.} \\
\hline
Null statement &
 4.4 &
 \syntax{$S \rightarrow$} &
 \example{} &
 \semantics{follow($S$) = start($S$).} \\
\hline
Conditional statement &
 5.1 &
 \syntax{$S_1 \rightarrow$ {\bfseries if the tape symbol is ``$I$'' then} $S_2$} &
 \example{{\bfseries if the tape symbol is ``$marilyn$'' then print ``$jayne$''}} &
 \semantics{{\bfseries define} $\delta$(start($S_1$),$s$) = ($s$, 0, follow($S_2$)) for all $s \in \Sigma -$ symbol(text($I$)); \\
	{\bfseries define} $\delta$(start($S_1$),$s$) = ($s$, 0, start($S_2$)) for $s =$ symbol(text($I$)); \\
	start($S_2$) = {\bfseries newsymbol;} \\
	follow($S_1$) = follow($S_2$); \\
	{\bfseries include} start($S_2$) {\bfseries in} $Q$.} \\
\hline
Labeled statement &
 5.2 &
 \syntax{$S_1 \rightarrow I : S_2$} &
 \example{$boston:$ {\bfseries move left one square}} &
 \semantics{{\bfseries define} label(text($I$)) = start($S_1$); \\
 	start($S_2$) = start($S_1$); \\
	follow($S_1$) = follow($S_2$).} \\
\hline
Compound statement &
 5.3 &
 \syntax{$S \rightarrow$ \{$L$\}} &
 \example{\{{\bfseries print} ``$jayne$'';; \\ {\bfseries go to} $boston$\}} &
 \semantics{start($L$) = start($S$); \\
	follow($S$) = follow($L$).} \\
\hline
List of statements &
 6.1 &
 \syntax{$L \rightarrow S$} &
 \example{{\bfseries print} ``$jayne$''} &
 \semantics{start($S$) = start($L$); \\
	follow($S$) = follow($L$).} \\
&
 6.2 &
 \syntax{$L_1 \rightarrow L_2$; $S$} &
 \example{{\bfseries print} ``$jayne$'';; {\bfseries go to} $boston$} &
 \semantics{start($L_2$) = start($L_1$); \\
	follow($L_2$) = {\bfseries newsymbol;} \\
	{\bfseries include} follow($L_2$) in $Q$; \\
	start($S$) = follow($L_2$); \\
	follow($L_1$) = follow($S$).} \\
\hline
Program &
 7 &
 \syntax{$P \rightarrow D$; $L$.} &
 \example{{\bfseries tape alphabet is} $marilyn, jayne, birgitta;$ {\bfseries print} ``$jayne$''.} &
 \semantics{$q_0$ = {\bfseries newsymbol;} \\
	{\bfseries include} $q_0$ {\bfseries in} $Q$; \\
	start($L$) = $q_0$; \\
	$q_\infty$ = follow($L$).}
\\
\hline
\end{tabular}
\end{center}

The method of defining semantic rules in this example, with an inherited
``first($S$)'' and a synthesized ``follow($S$)'' attribute, lends itself
readily also to computers or automata in which the $(n+1)$st instruction
normally is performed after the $n$-th. Then (follow($S$) $-$ start($S$)) would
be the number of instructions ``compiled'' for statement $S$.

This definition of Turingol seems to approach the desirable goal of stating
almost exactly the same things which would appear in an informal programmer's
manual explaining the language, except that the description is completely
formal and unambiguous. In other words, this definition perhaps corresponds to
the way we actually understand the language in our minds. The Definition 4.1 of
a print statement, for example, might be freely rendered in English as follow:

``A statement may have the form

\begin{center}
	{\bfseries print} ``$I$''
\end{center}
where $I$ is an identifier. This means that, whenever this statement is
executed, the tape symbol on the currently scanned square will be replaced by
the symbol denoted by $I$, regardless of what symbol was being scanned;
afterwards the program will continue with a new instruction, which is defined
(by other rules) to be the instruction following this statement.''

\end{document}

\input texinfo
@setfilename cpp.info
@settitle The C Preprocessor

@ignore
@ifinfo
@format
START-INFO-DIR-ENTRY
* Cpp: (cpp).			The C preprocessor.
END-INFO-DIR-ENTRY
@end format
@end ifinfo
@end ignore

@c @smallbook
@c @cropmarks
@c @finalout
@setchapternewpage odd
@ifinfo
This file documents the GNU C Preprocessor.

Copyright 1987, 1989, 1991, 1992, 1993 Free Software Foundation, Inc.

Permission is granted to make and distribute verbatim copies of
this manual provided the copyright notice and this permission notice
are preserved on all copies.

@ignore
Permission is granted to process this file through Tex and print the
results, provided the printed document carries copying permission
notice identical to this one except for the removal of this paragraph
(this paragraph not being relevant to the printed manual).

@end ignore
Permission is granted to copy and distribute modified versions of this
manual under the conditions for verbatim copying, provided also that
the entire resulting derived work is distributed under the terms of a
permission notice identical to this one.

Permission is granted to copy and distribute translations of this manual
into another language, under the above conditions for modified versions.
@end ifinfo

@titlepage
@c @finalout
@title The C Preprocessor
@subtitle Last revised July 1992
@subtitle for GCC version 2
@author Richard M. Stallman
@page
@vskip 2pc
This booklet is eventually intended to form the first chapter of a GNU 
C Language manual.

@vskip 0pt plus 1filll
Copyright @copyright{} 1987, 1989, 1991, 1992 Free Software Foundation, Inc.

Permission is granted to make and distribute verbatim copies of
this manual provided the copyright notice and this permission notice
are preserved on all copies.

Permission is granted to copy and distribute modified versions of this
manual under the conditions for verbatim copying, provided also that
the entire resulting derived work is distributed under the terms of a
permission notice identical to this one.

Permission is granted to copy and distribute translations of this manual
into another language, under the above conditions for modified versions.
@end titlepage
@page

@node Top, Global Actions,, (DIR)
@chapter The C Preprocessor

The C preprocessor is a @dfn{macro processor} that is used automatically by
the C compiler to transform your program before actual compilation.  It is
called a macro processor because it allows you to define @dfn{macros},
which are brief abbreviations for longer constructs.

The C preprocessor provides four separate facilities that you can use as
you see fit:

@itemize @bullet
@item
Inclusion of header files.  These are files of declarations that can be
substituted into your program.

@item
Macro expansion.  You can define @dfn{macros}, which are abbreviations
for arbitrary fragments of C code, and then the C preprocessor will
replace the macros with their definitions throughout the program.

@item
Conditional compilation.  Using special preprocessor commands, you
can include or exclude parts of the program according to various
conditions.

@item
Line control.  If you use a program to combine or rearrange source files into
an intermediate file which is then compiled, you can use line control
to inform the compiler of where each source line originally came from.
@end itemize

C preprocessors vary in some details.  This manual discusses the GNU C
preprocessor, the C Compatible Compiler Preprocessor.  The GNU C
preprocessor provides a superset of the features of ANSI Standard C.

ANSI Standard C requires the rejection of many harmless constructs commonly
used by today's C programs.  Such incompatibility would be inconvenient for
users, so the GNU C preprocessor is configured to accept these constructs
by default.  Strictly speaking, to get ANSI Standard C, you must use the
options @samp{-trigraphs}, @samp{-undef} and @samp{-pedantic}, but in
practice the consequences of having strict ANSI Standard C make it
undesirable to do this.  @xref{Invocation}.

@menu
* Global Actions::    Actions made uniformly on all input files.
* Commands::          General syntax of preprocessor commands.
* Header Files::      How and why to use header files.
* Macros::            How and why to use macros.
* Conditionals::      How and why to use conditionals.
* Combining Sources:: Use of line control when you combine source files.
* Other Commands::    Miscellaneous preprocessor commands.
* Output::            Format of output from the C preprocessor.
* Invocation::        How to invoke the preprocessor; command options.
* Concept Index::     Index of concepts and terms.
* Index::             Index of commands, predefined macros and options.
@end menu

@node Global Actions, Commands, Top, Top
@section Transformations Made Globally

Most C preprocessor features are inactive unless you give specific commands
to request their use.  (Preprocessor commands are lines starting with
@samp{#}; @pxref{Commands}).  But there are three transformations that the
preprocessor always makes on all the input it receives, even in the absence
of commands.

@itemize @bullet
@item
All C comments are replaced with single spaces.

@item
Backslash-Newline sequences are deleted, no matter where.  This
feature allows you to break long lines for cosmetic purposes without
changing their meaning.

@item
Predefined macro names are replaced with their expansions
(@pxref{Predefined}).
@end itemize

The first two transformations are done @emph{before} nearly all other parsing
and before preprocessor commands are recognized.  Thus, for example, you
can split a line cosmetically with Backslash-Newline anywhere (except
when trigraphs are in use; see below).

@example
/*
*/ # /*
*/ defi\
ne FO\
O 10\
20
@end example

@noindent
is equivalent into @samp{#define FOO 1020}.  You can split even an escape
sequence with Backslash-Newline.  For example, you can split @code{"foo\bar"}
between the @samp{\} and the @samp{b} to get

@example
"foo\\
bar"
@end example

@noindent
This behavior is unclean: in all other contexts, a Backslash can be
inserted in a string constant as an ordinary character by writing a double
Backslash, and this creates an exception.  But the ANSI C standard requires
it.  (Strict ANSI C does not allow Newlines in string constants, so they
do not consider this a problem.)

But there are a few exceptions to all three transformations.

@itemize @bullet
@item
C comments and predefined macro names are not recognized inside a
@samp{#include} command in which the file name is delimited with
@samp{<} and @samp{>}.

@item
C comments and predefined macro names are never recognized within a
character or string constant.  (Strictly speaking, this is the rule,
not an exception, but it is worth noting here anyway.)

@item
Backslash-Newline may not safely be used within an ANSI ``trigraph''.
Trigraphs are converted before Backslash-Newline is deleted.  If you
write what looks like a trigraph with a Backslash-Newline inside, the
Backslash-Newline is deleted as usual, but it is then too late to
recognize the trigraph.

This exception is relevant only if you use the @samp{-trigraphs}
option to enable trigraph processing.  @xref{Invocation}.
@end itemize

@node Commands, Header Files, Global Actions, Top
@section Preprocessor Commands

@cindex preprocessor commands
@cindex commands
Most preprocessor features are active only if you use preprocessor commands
to request their use.

Preprocessor commands are lines in your program that start with @samp{#}.
The @samp{#} is followed by an identifier that is the @dfn{command name}.
For example, @samp{#define} is the command that defines a macro.
Whitespace is also allowed before and after the @samp{#}.

The set of valid command names is fixed.  Programs cannot define new
preprocessor commands.

Some command names require arguments; these make up the rest of the command
line and must be separated from the command name by whitespace.  For example,
@samp{#define} must be followed by a macro name and the intended expansion
of the macro.

A preprocessor command cannot be more than one line in normal circumstances.
It may be split cosmetically with Backslash-Newline, but that has no effect
on its meaning.  Comments containing Newlines can also divide the command into
multiple lines, but the comments are changed to Spaces before the command
is interpreted.  The only way a significant Newline can occur in a preprocessor
command is within a string constant or character constant.  Note that
most C compilers that might be applied to the output from the preprocessor
do not accept string or character constants containing Newlines.

The @samp{#} and the command name cannot come from a macro expansion.  For
example, if @samp{foo} is defined as a macro expanding to @samp{define},
that does not make @samp{#foo} a valid preprocessor command.

@node Header Files, Macros, Commands, Top
@section Header Files

@cindex header file
A header file is a file containing C declarations and macro definitions
(@pxref{Macros}) to be shared between several source files.  You request
the use of a header file in your program with the C preprocessor command
@samp{#include}.

@menu
* Header Uses::         What header files are used for.
* Include Syntax::      How to write @samp{#include} commands.
* Include Operation::   What @samp{#include} does.
* Once-Only::		Preventing multiple inclusion of one header file.
* Inheritance::         Including one header file in another header file.
@end menu

@node Header Uses, Include Syntax, Header Files, Header Files
@subsection Uses of Header Files

Header files serve two kinds of purposes.

@itemize @bullet
@item
@findex system header files
System header files declare the interfaces to parts of the operating
system.  You include them in your program to supply the definitions and
declarations you need to invoke system calls and libraries.

@item
Your own header files contain declarations for interfaces between the
source files of your program.  Each time you have a group of related
declarations and macro definitions all or most of which are needed in
several different source files, it is a good idea to create a header
file for them.
@end itemize

Including a header file produces the same results in C compilation as
copying the header file into each source file that needs it.  But such
copying would be time-consuming and error-prone.  With a header file, the
related declarations appear in only one place.  If they need to be changed,
they can be changed in one place, and programs that include the header file
will automatically use the new version when next recompiled.  The header
file eliminates the labor of finding and changing all the copies as well as
the risk that a failure to find one copy will result in inconsistencies
within a program.

The usual convention is to give header files names that end with @file{.h}.

@node Include Syntax, Include Operation, Header Uses, Header Files
@subsection The @samp{#include} Command

@findex #include
Both user and system header files are included using the preprocessor
command @samp{#include}.  It has three variants:

@table @code
@item #include <@var{file}>
This variant is used for system header files.  It searches for a file
named @var{file} in a list of directories specified by you, then in a
standard list of system directories.  You specify directories to
search for header files with the command option @samp{-I}
(@pxref{Invocation}).  The option @samp{-nostdinc} inhibits searching
the standard system directories; in this case only the directories
you specify are searched.

The parsing of this form of @samp{#include} is slightly special
because comments are not recognized within the @samp{<@dots{}>}.
Thus, in @samp{#include <x/*y>} the @samp{/*} does not start a comment
and the command specifies inclusion of a system header file named
@file{x/*y}.  Of course, a header file with such a name is unlikely to
exist on Unix, where shell wildcard features would make it hard to
manipulate.@refill

The argument @var{file} may not contain a @samp{>} character.  It may,
however, contain a @samp{<} character.

@item #include "@var{file}"
This variant is used for header files of your own program.  It
searches for a file named @var{file} first in the current directory,
then in the same directories used for system header files.  The
current directory is the directory of the current input file.  It is
tried first because it is presumed to be the location of the files
that the current input file refers to.  (If the @samp{-I-} option is
used, the special treatment of the current directory is inhibited.)

The argument @var{file} may not contain @samp{"} characters.  If
backslashes occur within @var{file}, they are considered ordinary text
characters, not escape characters.  None of the character escape
sequences appropriate to string constants in C are processed.  Thus,
@samp{#include "x\n\\y"} specifies a filename containing three
backslashes.  It is not clear why this behavior is ever useful, but
the ANSI standard specifies it.

@item #include @var{anything else}
This variant is called a @dfn{computed #include}.  Any @samp{#include}
command whose argument does not fit the above two forms is a computed
include.  The text @var{anything else} is checked for macro calls,
which are expanded (@pxref{Macros}).  When this is done, the result
must fit one of the above two variants---in particular, the expanded
text must in the end be surrounded by either quotes or angle braces.

This feature allows you to define a macro which controls the file name
to be used at a later point in the program.  One application of this
is to allow a site-configuration file for your program to specify the
names of the system include files to be used.  This can help in
porting the program to various operating systems in which the
necessary system header files are found in different places.
@end table

@node Include Operation, Once-Only, Include Syntax, Header Files
@subsection How @samp{#include} Works

The @samp{#include} command works by directing the C preprocessor to scan
the specified file as input before continuing with the rest of the current
file.  The output from the preprocessor contains the output already
generated, followed by the output resulting from the included file,
followed by the output that comes from the text after the @samp{#include}
command.  For example, given two files as follows:

@example
/* File program.c */
int x;
#include "header.h"

main ()
@{
  printf (test ());
@}


/* File header.h */
char *test ();
@end example

@noindent
the output generated by the C preprocessor for @file{program.c} as input
would be

@example
int x;
char *test ();

main ()
@{
  printf (test ());
@}
@end example

Included files are not limited to declarations and macro definitions; those
are merely the typical uses.  Any fragment of a C program can be included
from another file.  The include file could even contain the beginning of a
statement that is concluded in the containing file, or the end of a
statement that was started in the including file.  However, a comment or a
string or character constant may not start in the included file and finish
in the including file.  An unterminated comment, string constant or
character constant in an included file is considered to end (with an error
message) at the end of the file.

The line following the @samp{#include} command is always treated as a
separate line by the C preprocessor even if the included file lacks a final
newline.

@node Once-Only, Inheritance, Include Operation, Header Files
@subsection Once-Only Include Files
@cindex repeated inclusion

Very often, one header file includes another.  It can easily result that a
certain header file is included more than once.  This may lead to errors,
if the header file defines structure types or typedefs, and is certainly
wasteful.  Therefore, we often wish to prevent multiple inclusion of a
header file.

The standard way to do this is to enclose the entire real contents of the
file in a conditional, like this:

@example
#ifndef __FILE_FOO_SEEN__
#define __FILE_FOO_SEEN__

@var{the entire file}

#endif /* __FILE_FOO_SEEN__ */
@end example

The macro @code{__FILE_FOO_SEEN__} indicates that the file has been
included once already; its name should begin with @samp{__} to avoid
conflicts with user programs, and it should contain the name of the file
and some additional text, to avoid conflicts with other header files.

The GNU C preprocessor is programmed to notice when a header file uses
this particular construct and handle it efficiently.  If a header file
is contained entirely in a @samp{#ifndef} conditional, then it records
that fact.  If a subsequent @samp{#include} specifies the same file,
and the macro in the @samp{#ifndef} is already defined, then the file
is entirely skipped, without even reading it.

@findex #pragma once
There is also an explicit command to tell the preprocessor that it need
not include a file more than once.  This is called @samp{#pragma once},
and was used @emph{in addition to} the @samp{#ifndef} conditional around
the contents of the header file.  @samp{#pragma once} is now obsolete
and should not be used at all.

In the Objective C language, there is a variant of @samp{#include}
called @samp{#import} which includes a file, but does so at most once.
If you use @samp{#import} @emph{instead of} @samp{#include}, then you
don't need the conditionals inside the header file to prevent multiple
execution of the contents.

@samp{#import} is obsolete because it is not a well-designed feature.
It requires the users of a header file---the applications
programmers---to know that a certain header file should only be included
once.  It is much better for the header file's implementor to write the
file so that users don't need to know this.  Using @samp{#ifndef}
accomplishes this goal.

@node Inheritance,, Once-Only, Header Files
@section Inheritance and Header Files
@cindex inheritance
@cindex overriding a header file

@dfn{Inheritance} is what happens when one object or file derives some
of its contents by virtual copying from another object or file.  In
the case of C header files, inheritance means that one header file 
includes another header file and then replaces or adds something.

If the inheriting header file and the base header file have different
names, then inheritance is straightforward: simply write @samp{#include
"@var{base}"} in the inheriting file.

Sometimes it is necessary to give the inheriting file the same name as
the base file.  This is less straightforward.

For example, suppose an application program uses the system header file
@file{sys/signal.h}, but the version of @file{/usr/include/sys/signal.h}
on a particular system doesn't do what the application program expects.
It might be convenient to define a ``local'' version, perhaps under the
name @file{/usr/local/include/sys/signal.h}, to override or add to the
one supplied by the system.

You can do this by using the option @samp{-I.} for compilation, and
writing a file @file{sys/signal.h} that does what the application
program expects.  But making this file include the standard
@file{sys/signal.h} is not so easy---writing @samp{#include
<sys/signal.h>} in that file doesn't work, because it includes your own
version of the file, not the standard system version.  Used in that file
itself, this leads to an infinite recursion and a fatal error in
compilation.

@samp{#include </usr/include/sys/signal.h>} would find the proper file,
but that is not clean, since it makes an assumption about where the
system header file is found.  This is bad for maintenance, since it
means that any change in where the system's header files are kept
requires a change somewhere else.

@findex #include_next
The clean way to solve this problem is to use 
@samp{#include_next}, which means, ``Include the @emph{next} file with
this name.''  This command works like @samp{#include} except in
searching for the specified file: it starts searching the list of header
file directories @emph{after} the directory in which the current file
was found.

Suppose you specify @samp{-I /usr/local/include}, and the list of
directories to search also includes @file{/usr/include}; and suppose that
both directories contain a file named @file{sys/signal.h}.  Ordinary
@samp{#include <sys/signal.h>} finds the file under
@file{/usr/local/include}.  If that file contains @samp{#include_next
<sys/signal.h>}, it starts searching after that directory, and finds the
file in @file{/usr/include}.

@node Macros, Conditionals, Header Files, Top
@section Macros

A macro is a sort of abbreviation which you can define once and then
use later.  There are many complicated features associated with macros
in the C preprocessor.

@menu
* Simple Macros::    Macros that always expand the same way.
* Argument Macros::  Macros that accept arguments that are substituted
                       into the macro expansion.
* Predefined::       Predefined macros that are always available.
* Stringification::  Macro arguments converted into string constants.
* Concatenation::    Building tokens from parts taken from macro arguments.
* Undefining::       Cancelling a macro's definition.
* Redefining::       Changing a macro's definition.
* Macro Pitfalls::   Macros can confuse the unwary.  Here we explain
                       several common problems and strange features.
@end menu

@node Simple Macros, Argument Macros, Macros, Macros
@subsection Simple Macros

A @dfn{simple macro} is a kind of abbreviation.  It is a name which
stands for a fragment of code.  Some people refer to these as
@dfn{manifest constants}.

Before you can use a macro, you must @dfn{define} it explicitly with the
@samp{#define} command.  @samp{#define} is followed by the name of the
macro and then the code it should be an abbreviation for.  For example,

@example
#define BUFFER_SIZE 1020
@end example

@noindent
defines a macro named @samp{BUFFER_SIZE} as an abbreviation for the text
@samp{1020}.  Therefore, if somewhere after this @samp{#define} command
there comes a C statement of the form

@example
foo = (char *) xmalloc (BUFFER_SIZE);
@end example

@noindent
then the C preprocessor will recognize and @dfn{expand} the macro
@samp{BUFFER_SIZE}, resulting in

@example
foo = (char *) xmalloc (1020);
@end example

@noindent
the definition must be a single line; however, it may not end in the
middle of a multi-line string constant or character constant.

The use of all upper case for macro names is a standard convention.
Programs are easier to read when it is possible to tell at a glance which
names are macros.

Normally, a macro definition must be a single line, like all C preprocessor
commands.  (You can split a long macro definition cosmetically with
Backslash-Newline.)  There is one exception: Newlines can be included in
the macro definition if within a string or character constant.  By the same
token, it is not possible for a macro definition to contain an unbalanced
quote character; the definition automatically extends to include the
matching quote character that ends the string or character constant.
Comments within a macro definition may contain Newlines, which make no
difference since the comments are entirely replaced with Spaces regardless
of their contents.

Aside from the above, there is no restriction on what can go in a macro
body.  Parentheses need not balance.  The body need not resemble valid C
code.  (Of course, you might get error messages from the C compiler when
you use the macro.)

The C preprocessor scans your program sequentially, so macro definitions
take effect at the place you write them.  Therefore, the following input to
the C preprocessor

@example
foo = X;
#define X 4
bar = X;
@end example

@noindent
produces as output

@example
foo = X;

bar = 4;
@end example

After the preprocessor expands a macro name, the macro's definition body is
appended to the front of the remaining input, and the check for macro calls
continues.  Therefore, the macro body can contain calls to other macros.
For example, after

@example
#define BUFSIZE 1020
#define TABLESIZE BUFSIZE
@end example

@noindent
the name @samp{TABLESIZE} when used in the program would go through two
stages of expansion, resulting ultimately in @samp{1020}.

This is not at all the same as defining @samp{TABLESIZE} to be @samp{1020}.
The @samp{#define} for @samp{TABLESIZE} uses exactly the body you
specify---in this case, @samp{BUFSIZE}---and does not check to see whether
it too is the name of a macro.  It's only when you @emph{use} @samp{TABLESIZE}
that the result of its expansion is checked for more macro names.
@xref{Cascaded Macros}.

@node Argument Macros, Predefined, Simple Macros, Macros
@subsection Macros with Arguments

A simple macro always stands for exactly the same text, each time it is
used.  Macros can be more flexible when they accept @dfn{arguments}.
Arguments are fragments of code that you supply each time the macro is
used.  These fragments are included in the expansion of the macro according
to the directions in the macro definition.

To define a macro that uses arguments, you write a @samp{#define} command
with a list of @dfn{argument names} in parentheses after the name of the
macro.  The argument names may be any valid C identifiers, separated by
commas and optionally whitespace.  The open-parenthesis must follow the
macro name immediately, with no space in between.

For example, here is a macro that computes the minimum of two numeric
values, as it is defined in many C programs:

@example
#define min(X, Y)  ((X) < (Y) ? (X) : (Y))
@end example

@noindent
(This is not the best way to define a ``minimum'' macro in GNU C.
@xref{Side Effects}, for more information.)

To use a macro that expects arguments, you write the name of the macro
followed by a list of @dfn{actual arguments} in parentheses, separated by
commas.  The number of actual arguments you give must match the number of
arguments the macro expects.   Examples of use of the macro @samp{min}
include @samp{min (1, 2)} and @samp{min (x + 28, *p)}.

The expansion text of the macro depends on the arguments you use.
Each of the argument names of the macro is replaced, throughout the
macro definition, with the corresponding actual argument.  Using the
same macro @samp{min} defined above, @samp{min (1, 2)} expands into

@example
((1) < (2) ? (1) : (2))
@end example

@noindent
where @samp{1} has been substituted for @samp{X} and @samp{2} for @samp{Y}.

Likewise, @samp{min (x + 28, *p)} expands into

@example
((x + 28) < (*p) ? (x + 28) : (*p))
@end example

Parentheses in the actual arguments must balance; a comma within
parentheses does not end an argument.  However, there is no requirement
for brackets or braces to balance, and they do not prevent a comma from
separating arguments.  Thus,

@example
macro (array[x = y, x + 1])
@end example

@noindent
passes two arguments to @code{macro}: @samp{array[x = y} and @samp{x +
1]}.  If you want to supply @samp{array[x = y, x + 1]} as an argument,
you must write it as @samp{array[(x = y, x + 1)]}, which is equivalent C
code.

After the actual arguments are substituted into the macro body, the entire
result is appended to the front of the remaining input, and the check for
macro calls continues.  Therefore, the actual arguments can contain calls
to other macros, either with or without arguments, or even to the same
macro.  The macro body can also contain calls to other macros.  For
example, @samp{min (min (a, b), c)} expands into this text:

@example
((((a) < (b) ? (a) : (b))) < (c)
 ? (((a) < (b) ? (a) : (b)))
 : (c))
@end example

@noindent
(Line breaks shown here for clarity would not actually be generated.)

If a macro @code{foo} takes one argument, and you want to supply an
empty argument, you must write at least some whitespace between the
parentheses, like this: @samp{foo ( )}.  Just @samp{foo ()} is providing
no arguments, which is an error if @code{foo} expects an argument.  But
@samp{foo0 ()} is the correct way to call a macro defined to take zero
arguments, like this:

@example
#define foo0() @dots{}
@end example

If you use the macro name followed by something other than an
open-parenthesis (after ignoring any spaces, tabs and comments that
follow), it is not a call to the macro, and the preprocessor does not
change what you have written.  Therefore, it is possible for the same name
to be a variable or function in your program as well as a macro, and you
can choose in each instance whether to refer to the macro (if an actual
argument list follows) or the variable or function (if an argument list
does not follow).

Such dual use of one name could be confusing and should be avoided
except when the two meanings are effectively synonymous: that is, when the
name is both a macro and a function and the two have similar effects.  You
can think of the name simply as a function; use of the name for purposes
other than calling it (such as, to take the address) will refer to the
function, while calls will expand the macro and generate better but
equivalent code.  For example, you can use a function named @samp{min} in
the same source file that defines the macro.  If you write @samp{&min} with
no argument list, you refer to the function.  If you write @samp{min (x,
bb)}, with an argument list, the macro is expanded.  If you write
@samp{(min) (a, bb)}, where the name @samp{min} is not followed by an
open-parenthesis, the macro is not expanded, so you wind up with a call to
the function @samp{min}.

You may not define the same name as both a simple macro and a macro with
arguments.

In the definition of a macro with arguments, the list of argument names
must follow the macro name immediately with no space in between.  If there
is a space after the macro name, the macro is defined as taking no
arguments, and all the rest of the line is taken to be the expansion.  The
reason for this is that it is often useful to define a macro that takes no
arguments and whose definition begins with an identifier in parentheses.
This rule about spaces makes it possible for you to do either this:

@example
#define FOO(x) - 1 / (x)
@end example

@noindent
(which defines @samp{FOO} to take an argument and expand into minus the
reciprocal of that argument) or this:

@example
#define BAR (x) - 1 / (x)
@end example

@noindent
(which defines @samp{BAR} to take no argument and always expand into
@samp{(x) - 1 / (x)}).

Note that the @emph{uses} of a macro with arguments can have spaces before
the left parenthesis; it's the @emph{definition} where it matters whether
there is a space.

@node Predefined, Stringification, Argument Macros, Macros
@subsection Predefined Macros

@cindex predefined macros
Several simple macros are predefined.  You can use them without giving
definitions for them.  They fall into two classes: standard macros and
system-specific macros.

@menu
* Standard Predefined::     Standard predefined macros.
* Nonstandard Predefined::  Nonstandard predefined macros.
@end menu

@node Standard Predefined, Nonstandard Predefined, Predefined, Predefined
@subsubsection Standard Predefined Macros

The standard predefined macros are available with the same meanings
regardless of the machine or operating system on which you are using GNU C.
Their names all start and end with double underscores.  Those preceding
@code{__GNUC__} in this table are standardized by ANSI C; the rest are
GNU C extensions.

@table @code
@item __FILE__
@findex __FILE__
This macro expands to the name of the current input file, in the form of
a C string constant.  The precise name returned is the one that was
specified in @samp{#include} or as the input file name argument.

@item __LINE__
@findex __LINE__
This macro expands to the current input line number, in the form of a
decimal integer constant.  While we call it a predefined macro, it's
a pretty strange macro, since its ``definition'' changes with each
new line of source code.

This and @samp{__FILE__} are useful in generating an error message to
report an inconsistency detected by the program; the message can state
the source line at which the inconsistency was detected.  For example,

@smallexample
fprintf (stderr, "Internal error: "
		 "negative string length "
                 "%d at %s, line %d.",
         length, __FILE__, __LINE__);
@end smallexample

A @samp{#include} command changes the expansions of @samp{__FILE__}
and @samp{__LINE__} to correspond to the included file.  At the end of
that file, when processing resumes on the input file that contained
the @samp{#include} command, the expansions of @samp{__FILE__} and
@samp{__LINE__} revert to the values they had before the
@samp{#include} (but @samp{__LINE__} is then incremented by one as
processing moves to the line after the @samp{#include}).

The expansions of both @samp{__FILE__} and @samp{__LINE__} are altered
if a @samp{#line} command is used.  @xref{Combining Sources}.

@item __INCLUDE_LEVEL__
@findex __INCLUDE_LEVEL_
This macro expands to a decimal integer constant that represents the
depth of nesting in include files.  The value of this macro is
incremented on every @samp{#include} command and decremented at every
end of file.

@item __DATE__
@findex __DATE__
This macro expands to a string constant that describes the date on
which the preprocessor is being run.  The string constant contains
eleven characters and looks like @samp{"Jan 29 1987"} or @w{@samp{"Apr
1 1905"}}.

@item __TIME__
@findex __TIME__
This macro expands to a string constant that describes the time at
which the preprocessor is being run.  The string constant contains
eight characters and looks like @samp{"23:59:01"}.

@item __STDC__
@findex __STDC__
This macro expands to the constant 1, to signify that this is ANSI
Standard C.  (Whether that is actually true depends on what C compiler
will operate on the output from the preprocessor.)

@item __GNUC__
This macro is defined if and only if this is GNU C.  This macro is
defined only when the entire GNU C compiler is in use; if you invoke
the preprocessor directly, @samp{__GNUC__} is undefined.

@item __GNUG__
The GNU C compiler defines this when the compilation language is
C++; use @samp{__GNUG__} to distinguish between GNU C and GNU
C++.

@item __cplusplus 
The draft ANSI standard for C++ used to require predefining this
variable.  Though it is no longer required, GNU C++ continues to define
it, as do other popular C++ compilers.  You can use @samp{__cplusplus}
to test whether a header is compiled by a C compiler or a C++ compiler.

@item __STRICT_ANSI__
This macro is defined if and only if the @samp{-ansi} switch was
specified when GNU C was invoked.  Its definition is the null string.
This macro exists primarily to direct certain GNU header files not to
define certain traditional Unix constructs which are incompatible with
ANSI C.

@item __BASE_FILE__
@findex __BASE_FILE__
This macro expands to the name of the main input file, in the form
of a C string constant.  This is the source file that was specified
as an argument when the C compiler was invoked.

@item __VERSION__
This macro expands to a string which describes the version number of
GNU C.  The string is normally a sequence of decimal numbers separated
by periods, such as @samp{"1.18"}.  The only reasonable use of this
macro is to incorporate it into a string constant.

@item __OPTIMIZE__
This macro is defined in optimizing compilations.  It causes certain
GNU header files to define alternative macro definitions for some
system library functions.  It is unwise to refer to or test the
definition of this macro unless you make very sure that programs will
execute with the same effect regardless.

@item __CHAR_UNSIGNED__
This macro is defined if and only if the data type @code{char} is
unsigned on the target machine.  It exists to cause the standard
header file @file{limit.h} to work correctly.  It is bad practice
to refer to this macro yourself; instead, refer to the standard
macros defined in @file{limit.h}.  The preprocessor uses
this macro to determine whether or not to sign-extend large character
constants written in octal; see @ref{#if Command,,The @samp{#if} Command}.
@end table

@node Nonstandard Predefined,, Standard Predefined, Predefined
@subsubsection Nonstandard Predefined Macros

The C preprocessor normally has several predefined macros that vary between
machines because their purpose is to indicate what type of system and
machine is in use.  This manual, being for all systems and machines, cannot
tell you exactly what their names are; instead, we offer a list of some
typical ones.  You can use @samp{cpp -dM} to see the values of
predefined macros; @pxref{Invocation}.

Some nonstandard predefined macros describe the operating system in use,
with more or less specificity.  For example,

@table @code
@item unix
@findex unix
@samp{unix} is normally predefined on all Unix systems.

@item BSD
@findex BSD
@samp{BSD} is predefined on recent versions of Berkeley Unix
(perhaps only in version 4.3).
@end table

Other nonstandard predefined macros describe the kind of CPU, with more or
less specificity.  For example,

@table @code
@item vax
@findex vax
@samp{vax} is predefined on Vax computers.

@item mc68000
@findex mc68000
@samp{mc68000} is predefined on most computers whose CPU is a Motorola
68000, 68010 or 68020.

@item m68k
@findex m68k
@samp{m68k} is also predefined on most computers whose CPU is a 68000,
68010 or 68020; however, some makers use @samp{mc68000} and some use
@samp{m68k}.  Some predefine both names.  What happens in GNU C
depends on the system you are using it on.

@item M68020
@findex M68020
@samp{M68020} has been observed to be predefined on some systems that
use 68020 CPUs---in addition to @samp{mc68000} and @samp{m68k}, which
are less specific.

@item _AM29K
@findex _AM29K
@itemx _AM29000
@findex _AM29000
Both @samp{_AM29K} and @samp{_AM29000} are predefined for the AMD 29000
CPU family.

@item ns32000
@findex ns32000
@samp{ns32000} is predefined on computers which use the National
Semiconductor 32000 series CPU.
@end table

Yet other nonstandard predefined macros describe the manufacturer of
the system.  For example,

@table @code
@item sun
@findex sun
@samp{sun} is predefined on all models of Sun computers.

@item pyr
@findex pyr
@samp{pyr} is predefined on all models of Pyramid computers.

@item sequent
@findex sequent
@samp{sequent} is predefined on all models of Sequent computers.
@end table

These predefined symbols are not only nonstandard, they are contrary to the
ANSI standard because their names do not start with underscores.
Therefore, the option @samp{-ansi} inhibits the definition of these
symbols.

This tends to make @samp{-ansi} useless, since many programs depend on the
customary nonstandard predefined symbols.  Even system header files check
them and will generate incorrect declarations if they do not find the names
that are expected.  You might think that the header files supplied for the
Uglix computer would not need to test what machine they are running on,
because they can simply assume it is the Uglix; but often they do, and they
do so using the customary names.  As a result, very few C programs will
compile with @samp{-ansi}.  We intend to avoid such problems on the GNU
system.

What, then, should you do in an ANSI C program to test the type of machine
it will run on?

GNU C offers a parallel series of symbols for this purpose, whose names
are made from the customary ones by adding @samp{__} at the beginning
and end.  Thus, the symbol @code{__vax__} would be available on a Vax,
and so on.

The set of nonstandard predefined names in the GNU C preprocessor is
controlled (when @code{cpp} is itself compiled) by the macro
@samp{CPP_PREDEFINES}, which should be a string containing @samp{-D}
options, separated by spaces.  For example, on the Sun 3, we use the
following definition:

@example
#define CPP_PREDEFINES "-Dmc68000 -Dsun -Dunix -Dm68k"
@end example

@noindent 
This macro is usually specified in @file{tm.h}.

@node Stringification, Concatenation, Predefined, Macros
@subsection Stringification

@cindex stringification
@dfn{Stringification} means turning a code fragment into a string constant
whose contents are the text for the code fragment.  For example,
stringifying @samp{foo (z)} results in @samp{"foo (z)"}.

In the C preprocessor, stringification is an option available when macro
arguments are substituted into the macro definition.  In the body of the
definition, when an argument name appears, the character @samp{#} before
the name specifies stringification of the corresponding actual argument
when it is substituted at that point in the definition.  The same argument
may be substituted in other places in the definition without
stringification if the argument name appears in those places with no
@samp{#}.

Here is an example of a macro definition that uses stringification:

@smallexample
#define WARN_IF(EXP) \
do @{ if (EXP) \
        fprintf (stderr, "Warning: " #EXP "\n"); @} \
while (0)
@end smallexample

@noindent
Here the actual argument for @samp{EXP} is substituted once as given,
into the @samp{if} statement, and once as stringified, into the
argument to @samp{fprintf}.  The @samp{do} and @samp{while (0)} are
a kludge to make it possible to write @samp{WARN_IF (@var{arg});},
which the resemblance of @samp{WARN_IF} to a function would make
C programmers want to do; @pxref{Swallow Semicolon}).

The stringification feature is limited to transforming one macro argument
into one string constant: there is no way to combine the argument with
other text and then stringify it all together.  But the example above shows
how an equivalent result can be obtained in ANSI Standard C using the
feature that adjacent string constants are concatenated as one string
constant.  The preprocessor stringifies the actual value of @samp{EXP} 
into a separate string constant, resulting in text like

@smallexample
do @{ if (x == 0) \
        fprintf (stderr, "Warning: " "x == 0" "\n"); @} \
while (0)
@end smallexample

@noindent
but the C compiler then sees three consecutive string constants and
concatenates them into one, producing effectively

@smallexample
do @{ if (x == 0) \
        fprintf (stderr, "Warning: x == 0\n"); @} \
while (0)
@end smallexample

Stringification in C involves more than putting doublequote characters
around the fragment; it is necessary to put backslashes in front of all
doublequote characters, and all backslashes in string and character
constants, in order to get a valid C string constant with the proper
contents.  Thus, stringifying @samp{p = "foo\n";} results in @samp{"p =
\"foo\\n\";"}.  However, backslashes that are not inside of string or
character constants are not duplicated: @samp{\n} by itself stringifies to
@samp{"\n"}.

Whitespace (including comments) in the text being stringified is handled
according to precise rules.  All leading and trailing whitespace is ignored.
Any sequence of whitespace in the middle of the text is converted to
a single space in the stringified result.

@node Concatenation, Undefining, Stringification, Macros
@subsection Concatenation

@cindex concatenation
@dfn{Concatenation} means joining two strings into one.  In the context
of macro expansion, concatenation refers to joining two lexical units
into one longer one.  Specifically, an actual argument to the macro can be
concatenated with another actual argument or with fixed text to produce
a longer name.  The longer name might be the name of a function,
variable or type, or a C keyword; it might even be the name of another
macro, in which case it will be expanded.

When you define a macro, you request concatenation with the special
operator @samp{##} in the macro body.  When the macro is called,
after actual arguments are substituted, all @samp{##} operators are
deleted, and so is any whitespace next to them (including whitespace
that was part of an actual argument).  The result is to concatenate
the syntactic tokens on either side of the @samp{##}.

Consider a C program that interprets named commands.  There probably needs
to be a table of commands, perhaps an array of structures declared as
follows:

@example
struct command
@{
  char *name;
  void (*function) ();
@};

struct command commands[] =
@{
  @{ "quit", quit_command@},
  @{ "help", help_command@},
  @dots{}
@};
@end example

It would be cleaner not to have to give each command name twice, once in
the string constant and once in the function name.  A macro which takes the
name of a command as an argument can make this unnecessary.  The string
constant can be created with stringification, and the function name by
concatenating the argument with @samp{_command}.  Here is how it is done:

@example
#define COMMAND(NAME)  @{ #NAME, NAME ## _command @}

struct command commands[] =
@{
  COMMAND (quit),
  COMMAND (help),
  @dots{}
@};
@end example

The usual case of concatenation is concatenating two names (or a name and a
number) into a longer name.  But this isn't the only valid case.  It is
also possible to concatenate two numbers (or a number and a name, such as
@samp{1.5} and @samp{e3}) into a number.  Also, multi-character operators
such as @samp{+=} can be formed by concatenation.  In some cases it is even
possible to piece together a string constant.  However, two pieces of text
that don't together form a valid lexical unit cannot be concatenated.  For
example, concatenation with @samp{x} on one side and @samp{+} on the other
is not meaningful because those two characters can't fit together in any
lexical unit of C.  The ANSI standard says that such attempts at
concatenation are undefined, but in the GNU C preprocessor it is well
defined: it puts the @samp{x} and @samp{+} side by side with no particular
special results.

Keep in mind that the C preprocessor converts comments to whitespace before
macros are even considered.  Therefore, you cannot create a comment by
concatenating @samp{/} and @samp{*}: the @samp{/*} sequence that starts a
comment is not a lexical unit, but rather the beginning of a ``long'' space
character.  Also, you can freely use comments next to a @samp{##} in a
macro definition, or in actual arguments that will be concatenated, because
the comments will be converted to spaces at first sight, and concatenation
will later discard the spaces.

@node Undefining, Redefining, Concatenation, Macros
@subsection Undefining Macros

@cindex undefining macros
To @dfn{undefine} a macro means to cancel its definition.  This is done
with the @samp{#undef} command.  @samp{#undef} is followed by the macro
name to be undefined.

Like definition, undefinition occurs at a specific point in the source
file, and it applies starting from that point.  The name ceases to be a
macro name, and from that point on it is treated by the preprocessor as if
it had never been a macro name.

For example,

@example
#define FOO 4
x = FOO;
#undef FOO
x = FOO;
@end example

@noindent
expands into

@example
x = 4;

x = FOO;
@end example

@noindent
In this example, @samp{FOO} had better be a variable or function as well
as (temporarily) a macro, in order for the result of the expansion to be
valid C code.

The same form of @samp{#undef} command will cancel definitions with
arguments or definitions that don't expect arguments.  The @samp{#undef}
command has no effect when used on a name not currently defined as a macro.

@node Redefining, Macro Pitfalls, Undefining, Macros
@subsection Redefining Macros

@cindex redefining macros
@dfn{Redefining} a macro means defining (with @samp{#define}) a name that
is already defined as a macro.

A redefinition is trivial if the new definition is transparently identical
to the old one.  You probably wouldn't deliberately write a trivial
redefinition, but they can happen automatically when a header file is
included more than once (@pxref{Header Files}), so they are accepted
silently and without effect.

Nontrivial redefinition is considered likely to be an error, so
it provokes a warning message from the preprocessor.  However, sometimes it
is useful to change the definition of a macro in mid-compilation.  You can
inhibit the warning by undefining the macro with @samp{#undef} before the
second definition.

In order for a redefinition to be trivial, the new definition must
exactly match the one already in effect, with two possible exceptions:

@itemize @bullet
@item
Whitespace may be added or deleted at the beginning or the end.

@item
Whitespace may be changed in the middle (but not inside strings).
However, it may not be eliminated entirely, and it may not be added
where there was no whitespace at all.
@end itemize

Recall that a comment counts as whitespace.

@node Macro Pitfalls,, Redefining, Macros
@subsection Pitfalls and Subtleties of Macros

In this section we describe some special rules that apply to macros and
macro expansion, and point out certain cases in which the rules have
counterintuitive consequences that you must watch out for.

@menu
* Misnesting::        Macros can contain unmatched parentheses.
* Macro Parentheses:: Why apparently superfluous parentheses
                         may be necessary to avoid incorrect grouping.
* Swallow Semicolon:: Macros that look like functions
                         but expand into compound statements.
* Side Effects::      Unsafe macros that cause trouble when
                         arguments contain side effects.
* Self-Reference::    Macros whose definitions use the macros' own names.
* Argument Prescan::  Actual arguments are checked for macro calls
                         before they are substituted.
* Cascaded Macros::   Macros whose definitions use other macros.
* Newlines in Args::  Sometimes line numbers get confused.
@end menu

@node Misnesting, Macro Parentheses, Macro Pitfalls, Macro Pitfalls
@subsubsection Improperly Nested Constructs

Recall that when a macro is called with arguments, the arguments are
substituted into the macro body and the result is checked, together with
the rest of the input file, for more macro calls.

It is possible to piece together a macro call coming partially from the
macro body and partially from the actual arguments.  For example,

@example
#define double(x) (2*(x))
#define call_with_1(x) x(1)
@end example

@noindent
would expand @samp{call_with_1 (double)} into @samp{(2*(1))}.

Macro definitions do not have to have balanced parentheses.  By writing an
unbalanced open parenthesis in a macro body, it is possible to create a
macro call that begins inside the macro body but ends outside of it.  For
example,

@example
#define strange(file) fprintf (file, "%s %d",
@dots{}
strange(stderr) p, 35)
@end example

@noindent
This bizarre example expands to @samp{fprintf (stderr, "%s %d", p, 35)}!

@node Macro Parentheses, Swallow Semicolon, Misnesting, Macro Pitfalls
@subsubsection Unintended Grouping of Arithmetic

You may have noticed that in most of the macro definition examples shown
above, each occurrence of a macro argument name had parentheses around it.
In addition, another pair of parentheses usually surround the entire macro
definition.  Here is why it is best to write macros that way.

Suppose you define a macro as follows,

@example
#define ceil_div(x, y) (x + y - 1) / y
@end example

@noindent
whose purpose is to divide, rounding up.  (One use for this operation is
to compute how many @samp{int} objects are needed to hold a certain
number of @samp{char} objects.)  Then suppose it is used as follows:

@example
a = ceil_div (b & c, sizeof (int));
@end example

@noindent
This expands into

@example
a = (b & c + sizeof (int) - 1) / sizeof (int);
@end example

@noindent
which does not do what is intended.  The operator-precedence rules of
C make it equivalent to this:

@example
a = (b & (c + sizeof (int) - 1)) / sizeof (int);
@end example

@noindent
But what we want is this:

@example
a = ((b & c) + sizeof (int) - 1)) / sizeof (int);
@end example

@noindent
Defining the macro as

@example
#define ceil_div(x, y) ((x) + (y) - 1) / (y)
@end example

@noindent
provides the desired result.

However, unintended grouping can result in another way.  Consider
@samp{sizeof ceil_div(1, 2)}.  That has the appearance of a C expression
that would compute the size of the type of @samp{ceil_div (1, 2)}, but in
fact it means something very different.  Here is what it expands to:

@example
sizeof ((1) + (2) - 1) / (2)
@end example

@noindent
This would take the size of an integer and divide it by two.  The precedence
rules have put the division outside the @samp{sizeof} when it was intended
to be inside.

Parentheses around the entire macro definition can prevent such problems.
Here, then, is the recommended way to define @samp{ceil_div}:

@example
#define ceil_div(x, y) (((x) + (y) - 1) / (y))
@end example

@node Swallow Semicolon, Side Effects, Macro Parentheses, Macro Pitfalls
@subsubsection Swallowing the Semicolon

@cindex semicolons (after macro calls)
Often it is desirable to define a macro that expands into a compound
statement.  Consider, for example, the following macro, that advances a
pointer (the argument @samp{p} says where to find it) across whitespace
characters:

@example
#define SKIP_SPACES (p, limit)  \
@{ register char *lim = (limit); \
  while (p != lim) @{            \
    if (*p++ != ' ') @{          \
      p--; break; @}@}@}
@end example

@noindent
Here Backslash-Newline is used to split the macro definition, which must
be a single line, so that it resembles the way such C code would be
laid out if not part of a macro definition.

A call to this macro might be @samp{SKIP_SPACES (p, lim)}.  Strictly
speaking, the call expands to a compound statement, which is a complete
statement with no need for a semicolon to end it.  But it looks like a
function call.  So it minimizes confusion if you can use it like a function
call, writing a semicolon afterward, as in @samp{SKIP_SPACES (p, lim);}

But this can cause trouble before @samp{else} statements, because the
semicolon is actually a null statement.  Suppose you write

@example
if (*p != 0)
  SKIP_SPACES (p, lim);
else @dots{}
@end example

@noindent
The presence of two statements---the compound statement and a null
statement---in between the @samp{if} condition and the @samp{else}
makes invalid C code.

The definition of the macro @samp{SKIP_SPACES} can be altered to solve
this problem, using a @samp{do @dots{} while} statement.  Here is how:

@example
#define SKIP_SPACES (p, limit)     \
do @{ register char *lim = (limit); \
     while (p != lim) @{            \
       if (*p++ != ' ') @{          \
         p--; break; @}@}@}           \
while (0)
@end example

Now @samp{SKIP_SPACES (p, lim);} expands into

@example
do @{@dots{}@} while (0);
@end example

@noindent
which is one statement.

@node Side Effects, Self-Reference, Swallow Semicolon, Macro Pitfalls
@subsubsection Duplication of Side Effects

@cindex side effects (in macro arguments)
@cindex unsafe macros
Many C programs define a macro @samp{min}, for ``minimum'', like this:

@example
#define min(X, Y)  ((X) < (Y) ? (X) : (Y))
@end example

When you use this macro with an argument containing a side effect,
as shown here,

@example
next = min (x + y, foo (z));
@end example

@noindent
it expands as follows:

@example
next = ((x + y) < (foo (z)) ? (x + y) : (foo (z)));
@end example

@noindent
where @samp{x + y} has been substituted for @samp{X} and @samp{foo (z)}
for @samp{Y}.

The function @samp{foo} is used only once in the statement as it appears
in the program, but the expression @samp{foo (z)} has been substituted
twice into the macro expansion.  As a result, @samp{foo} might be called
two times when the statement is executed.  If it has side effects or
if it takes a long time to compute, the results might not be what you
intended.  We say that @samp{min} is an @dfn{unsafe} macro.

The best solution to this problem is to define @samp{min} in a way that
computes the value of @samp{foo (z)} only once.  The C language offers no
standard way to do this, but it can be done with GNU C extensions as
follows:

@example
#define min(X, Y)                     \
(@{ typeof (X) __x = (X), __y = (Y);   \
   (__x < __y) ? __x : __y; @})
@end example

If you do not wish to use GNU C extensions, the only solution is to be
careful when @emph{using} the macro @samp{min}.  For example, you can
calculate the value of @samp{foo (z)}, save it in a variable, and use that
variable in @samp{min}:

@example
#define min(X, Y)  ((X) < (Y) ? (X) : (Y))
@dots{}
@{
  int tem = foo (z);
  next = min (x + y, tem);
@}
@end example

@noindent
(where we assume that @samp{foo} returns type @samp{int}).

@node Self-Reference, Argument Prescan, Side Effects, Macro Pitfalls
@subsubsection Self-Referential Macros

@cindex self-reference
A @dfn{self-referential} macro is one whose name appears in its definition.
A special feature of ANSI Standard C is that the self-reference is not
considered a macro call.  It is passed into the preprocessor output
unchanged.

Let's consider an example:

@example
#define foo (4 + foo)
@end example

@noindent
where @samp{foo} is also a variable in your program.

Following the ordinary rules, each reference to @samp{foo} will expand into
@samp{(4 + foo)}; then this will be rescanned and will expand into @samp{(4
+ (4 + foo))}; and so on until it causes a fatal error (memory full) in the
preprocessor.

However, the special rule about self-reference cuts this process short
after one step, at @samp{(4 + foo)}.  Therefore, this macro definition
has the possibly useful effect of causing the program to add 4 to
the value of @samp{foo} wherever @samp{foo} is referred to.

In most cases, it is a bad idea to take advantage of this feature.  A
person reading the program who sees that @samp{foo} is a variable will
not expect that it is a macro as well.  The reader will come across the
identifier @samp{foo} in the program and think its value should be that
of the variable @samp{foo}, whereas in fact the value is four greater.

The special rule for self-reference applies also to @dfn{indirect}
self-reference.  This is the case where a macro @var{x} expands to use a
macro @samp{y}, and the expansion of @samp{y} refers to the macro
@samp{x}.  The resulting reference to @samp{x} comes indirectly from the
expansion of @samp{x}, so it is a self-reference and is not further
expanded.  Thus, after

@example
#define x (4 + y)
#define y (2 * x)
@end example

@noindent
@samp{x} would expand into @samp{(4 + (2 * x))}.  Clear?

But suppose @samp{y} is used elsewhere, not from the definition of @samp{x}.
Then the use of @samp{x} in the expansion of @samp{y} is not a self-reference
because @samp{x} is not ``in progress''.  So it does expand.  However,
the expansion of @samp{x} contains a reference to @samp{y}, and that
is an indirect self-reference now because @samp{y} is ``in progress''.
The result is that @samp{y} expands to @samp{(2 * (4 + y))}.

It is not clear that this behavior would ever be useful, but it is specified
by the ANSI C standard, so you may need to understand it.

@node Argument Prescan, Cascaded Macros, Self-Reference, Macro Pitfalls
@subsubsection Separate Expansion of Macro Arguments

We have explained that the expansion of a macro, including the substituted
actual arguments, is scanned over again for macro calls to be expanded.

What really happens is more subtle: first each actual argument text is scanned
separately for macro calls.  Then the results of this are substituted into
the macro body to produce the macro expansion, and the macro expansion
is scanned again for macros to expand.

The result is that the actual arguments are scanned @emph{twice} to expand
macro calls in them.

Most of the time, this has no effect.  If the actual argument contained
any macro calls, they are expanded during the first scan.  The result
therefore contains no macro calls, so the second scan does not change it.
If the actual argument were substituted as given, with no prescan,
the single remaining scan would find the same macro calls and produce
the same results.

You might expect the double scan to change the results when a
self-referential macro is used in an actual argument of another macro
(@pxref{Self-Reference}): the self-referential macro would be expanded once
in the first scan, and a second time in the second scan.  But this is not
what happens.  The self-references that do not expand in the first scan are
marked so that they will not expand in the second scan either.

The prescan is not done when an argument is stringified or concatenated.
Thus,

@example
#define str(s) #s
#define foo 4
str (foo)
@end example

@noindent
expands to @samp{"foo"}.  Once more, prescan has been prevented from
having any noticeable effect.

More precisely, stringification and concatenation use the argument as
written, in un-prescanned form.  The same actual argument would be used in
prescanned form if it is substituted elsewhere without stringification or
concatenation.

@example
#define str(s) #s lose(s)
#define foo 4
str (foo)
@end example

expands to @samp{"foo" lose(4)}.

You might now ask, ``Why mention the prescan, if it makes no difference?
And why not skip it and make the preprocessor faster?''  The answer is
that the prescan does make a difference in three special cases:

@itemize @bullet
@item
Nested calls to a macro.

@item
Macros that call other macros that stringify or concatenate.

@item
Macros whose expansions contain unshielded commas.
@end itemize

We say that @dfn{nested} calls to a macro occur when a macro's actual
argument contains a call to that very macro.  For example, if @samp{f}
is a macro that expects one argument, @samp{f (f (1))} is a nested
pair of calls to @samp{f}.  The desired expansion is made by
expanding @samp{f (1)} and substituting that into the definition of
@samp{f}.  The prescan causes the expected result to happen.
Without the prescan, @samp{f (1)} itself would be substituted as
an actual argument, and the inner use of @samp{f} would appear
during the main scan as an indirect self-reference and would not
be expanded.  Here, the prescan cancels an undesirable side effect
(in the medical, not computational, sense of the term) of the special
rule for self-referential macros.

But prescan causes trouble in certain other cases of nested macro calls.
Here is an example:

@example
#define foo  a,b
#define bar(x) lose(x)
#define lose(x) (1 + (x))

bar(foo)
@end example

@noindent
We would like @samp{bar(foo)} to turn into @samp{(1 + (foo))}, which
would then turn into @samp{(1 + (a,b))}.  But instead, @samp{bar(foo)}
expands into @samp{lose(a,b)}, and you get an error because @code{lose}
requires a single argument.  In this case, the problem is easily solved
by the same parentheses that ought to be used to prevent misnesting of
arithmetic operations:

@example
#define foo (a,b)
#define bar(x) lose((x))
@end example

The problem is more serious when the operands of the macro are not
expressions; for example, when they are statements.  Then parentheses
are unacceptable because they would make for invalid C code:

@example
#define foo @{ int a, b; @dots{} @}
@end example

@noindent
In GNU C you can shield the commas using the @samp{(@{@dots{}@})}
construct which turns a compound statement into an expression:

@example
#define foo (@{ int a, b; @dots{} @})
@end example

Or you can rewrite the macro definition to avoid such commas:

@example
#define foo @{ int a; int b; @dots{} @}
@end example

There is also one case where prescan is useful.  It is possible
to use prescan to expand an argument and then stringify it---if you use
two levels of macros.  Let's add a new macro @samp{xstr} to the
example shown above:

@example
#define xstr(s) str(s)
#define str(s) #s
#define foo 4
xstr (foo)
@end example

This expands into @samp{"4"}, not @samp{"foo"}.  The reason for the
difference is that the argument of @samp{xstr} is expanded at prescan
(because @samp{xstr} does not specify stringification or concatenation of
the argument).  The result of prescan then forms the actual argument for
@samp{str}.  @samp{str} uses its argument without prescan because it
performs stringification; but it cannot prevent or undo the prescanning
already done by @samp{xstr}.

@node Cascaded Macros, Newlines in Args, Argument Prescan, Macro Pitfalls
@subsubsection Cascaded Use of Macros

@cindex cascaded macros
@cindex macro body uses macro
A @dfn{cascade} of macros is when one macro's body contains a reference
to another macro.  This is very common practice.  For example,

@example
#define BUFSIZE 1020
#define TABLESIZE BUFSIZE
@end example

This is not at all the same as defining @samp{TABLESIZE} to be @samp{1020}.
The @samp{#define} for @samp{TABLESIZE} uses exactly the body you
specify---in this case, @samp{BUFSIZE}---and does not check to see whether
it too is the name of a macro.

It's only when you @emph{use} @samp{TABLESIZE} that the result of its expansion
is checked for more macro names.

This makes a difference if you change the definition of @samp{BUFSIZE}
at some point in the source file.  @samp{TABLESIZE}, defined as shown,
will always expand using the definition of @samp{BUFSIZE} that is
currently in effect:

@example
#define BUFSIZE 1020
#define TABLESIZE BUFSIZE
#undef BUFSIZE
#define BUFSIZE 37
@end example

@noindent
Now @samp{TABLESIZE} expands (in two stages) to @samp{37}.

@node Newlines in Args,, Cascaded Macros, Macro Pitfalls
@subsection Newlines in Macro Arguments

Traditional macro processing carries forward all newlines in macro
arguments into the expansion of the macro.  This means that, if some of
the arguments are substituted more than once, or not at all, or out of
order, newlines can be duplicated, lost, or moved around within the
expansion.  If the expansion consists of multiple statements, then the
effect is to distort the line numbers of some of these statements.  The
result can be incorrect line numbers, in error messages or displayed in
a debugger.

The GNU C preprocessor operating in ANSI C mode adjusts appropriately
for multiple use of an argument---the first use expands all the
newlines, and subsequent uses of the same argument produce no newlines.
But even in this mode, it can produce incorrect line numbering if
arguments are used out of order, or not used at all.

Here is an example illustrating this problem:

@example
#define ignore_second_arg(a,b,c) a; c

ignore_second_arg (foo (),
                   ignored (),
                   syntax error);
@end example

@noindent
The syntax error triggered by the tokens @samp{syntax error} results
in an error message citing line four, even though the statement text
comes from line five.

@node Conditionals, Combining Sources, Macros, Top
@section Conditionals

@cindex conditionals
In a macro processor, a @dfn{conditional} is a command that allows a part
of the program to be ignored during compilation, on some conditions.
In the C preprocessor, a conditional can test either an arithmetic expression
or whether a name is defined as a macro.

A conditional in the C preprocessor resembles in some ways an @samp{if}
statement in C, but it is important to understand the difference between
them.  The condition in an @samp{if} statement is tested during the execution
of your program.  Its purpose is to allow your program to behave differently
from run to run, depending on the data it is operating on.  The condition
in a preprocessor conditional command is tested when your program is compiled.
Its purpose is to allow different code to be included in the program depending
on the situation at the time of compilation.

@menu
* Uses: Conditional Uses.       What conditionals are for.
* Syntax: Conditional Syntax.   How conditionals are written.
* Deletion: Deleted Code.       Making code into a comment.
* Macros: Conditionals-Macros.  Why conditionals are used with macros.
* Assertions::		        How and why to use assertions.
* Errors: #error Command.       Detecting inconsistent compilation parameters.
@end menu

@node Conditional Uses
@subsection Why Conditionals are Used

Generally there are three kinds of reason to use a conditional.

@itemize @bullet
@item
A program may need to use different code depending on the machine or
operating system it is to run on.  In some cases the code for one
operating system may be erroneous on another operating system; for
example, it might refer to library routines that do not exist on the
other system.  When this happens, it is not enough to avoid executing
the invalid code: merely having it in the program makes it impossible
to link the program and run it.  With a preprocessor conditional, the
offending code can be effectively excised from the program when it is
not valid.

@item
You may want to be able to compile the same source file into two
different programs.  Sometimes the difference between the programs is
that one makes frequent time-consuming consistency checks on its
intermediate data while the other does not.

@item
A conditional whose condition is always false is a good way to exclude
code from the program but keep it as a sort of comment for future
reference.
@end itemize

Most simple programs that are intended to run on only one machine will
not need to use preprocessor conditionals.

@node Conditional Syntax
@subsection Syntax of Conditionals

@findex #if
A conditional in the C preprocessor begins with a @dfn{conditional
command}: @samp{#if}, @samp{#ifdef} or @samp{#ifndef}.
@xref{Conditionals-Macros}, for information on @samp{#ifdef} and
@samp{#ifndef}; only @samp{#if} is explained here.

@menu
* If: #if Command.     Basic conditionals using @samp{#if} and @samp{#endif}.
* Else: #else Command. Including some text if the condition fails.
* Elif: #elif Command. Testing several alternative possibilities.
@end menu

@node #if Command
@subsubsection The @samp{#if} Command

The @samp{#if} command in its simplest form consists of

@example
#if @var{expression}
@var{controlled text}
#endif /* @var{expression} */
@end example

The comment following the @samp{#endif} is not required, but it is a good
practice because it helps people match the @samp{#endif} to the
corresponding @samp{#if}.  Such comments should always be used, except in
short conditionals that are not nested.  In fact, you can put anything at
all after the @samp{#endif} and it will be ignored by the GNU C preprocessor,
but only comments are acceptable in ANSI Standard C.

@var{expression} is a C expression of integer type, subject to stringent
restrictions.  It may contain

@itemize @bullet
@item
Integer constants, which are all regarded as @code{long} or
@code{unsigned long}.

@item
Character constants, which are interpreted according to the character
set and conventions of the machine and operating system on which the
preprocessor is running.  The GNU C preprocessor uses the C data type
@samp{char} for these character constants; therefore, whether some
character codes are negative is determined by the C compiler used to
compile the preprocessor.  If it treats @samp{char} as signed, then
character codes large enough to set the sign bit will be considered
negative; otherwise, no character code is considered negative.

@item
Arithmetic operators for addition, subtraction, multiplication,
division, bitwise operations, shifts, comparisons, and @samp{&&} and
@samp{||}.

@item
Identifiers that are not macros, which are all treated as zero(!).

@item
Macro calls.  All macro calls in the expression are expanded before
actual computation of the expression's value begins.
@end itemize

Note that @samp{sizeof} operators and @code{enum}-type values are not allowed.
@code{enum}-type values, like all other identifiers that are not taken
as macro calls and expanded, are treated as zero.

The @var{controlled text} inside of a conditional can include
preprocessor commands.  Then the commands inside the conditional are
obeyed only if that branch of the conditional succeeds.  The text can
also contain other conditional groups.  However, the @samp{#if} and
@samp{#endif} commands must balance.

@node #else Command
@subsubsection The @samp{#else} Command

@findex #else
The @samp{#else} command can be added to a conditional to provide
alternative text to be used if the condition is false.  This is what
it looks like:

@example
#if @var{expression}
@var{text-if-true}
#else /* Not @var{expression} */
@var{text-if-false}
#endif /* Not @var{expression} */
@end example

If @var{expression} is nonzero, and thus the @var{text-if-true} is 
active, then @samp{#else} acts like a failing conditional and the
@var{text-if-false} is ignored.  Contrariwise, if the @samp{#if}
conditional fails, the @var{text-if-false} is considered included.

@node #elif Command
@subsubsection The @samp{#elif} Command

@findex #elif
One common case of nested conditionals is used to check for more than two
possible alternatives.  For example, you might have

@example
#if X == 1
@dots{}
#else /* X != 1 */
#if X == 2
@dots{}
#else /* X != 2 */
@dots{}
#endif /* X != 2 */
#endif /* X != 1 */
@end example

Another conditional command, @samp{#elif}, allows this to be abbreviated
as follows:

@example
#if X == 1
@dots{}
#elif X == 2
@dots{}
#else /* X != 2 and X != 1*/
@dots{}
#endif /* X != 2 and X != 1*/
@end example

@samp{#elif} stands for ``else if''.  Like @samp{#else}, it goes in the
middle of a @samp{#if}-@samp{#endif} pair and subdivides it; it does not
require a matching @samp{#endif} of its own.  Like @samp{#if}, the
@samp{#elif} command includes an expression to be tested.

The text following the @samp{#elif} is processed only if the original
@samp{#if}-condition failed and the @samp{#elif} condition succeeds.
More than one @samp{#elif} can go in the same @samp{#if}-@samp{#endif}
group.  Then the text after each @samp{#elif} is processed only if the
@samp{#elif} condition succeeds after the original @samp{#if} and any
previous @samp{#elif} commands within it have failed.  @samp{#else} is
equivalent to @samp{#elif 1}, and @samp{#else} is allowed after any
number of @samp{#elif} commands, but @samp{#elif} may not follow
@samp{#else}.

@node Deleted Code
@subsection Keeping Deleted Code for Future Reference

If you replace or delete a part of the program but want to keep the old
code around as a comment for future reference, the easy way to do this is
to put @samp{#if 0} before it and @samp{#endif} after it.

This works even if the code being turned off contains conditionals, but
they must be entire conditionals (balanced @samp{#if} and @samp{#endif}).

@node Conditionals-Macros
@subsection Conditionals and Macros

Conditionals are useful in connection with macros or assertions, because
those are the only ways that an expression's value can vary from one
compilation to another.  A @samp{#if} command whose expression uses no
macros or assertions is equivalent to @samp{#if 1} or @samp{#if 0}; you
might as well determine which one, by computing the value of the
expression yourself, and then simplify the program.

For example, here is a conditional that tests the expression
@samp{BUFSIZE == 1020}, where @samp{BUFSIZE} must be a macro.

@example
#if BUFSIZE == 1020
  printf ("Large buffers!\n");
#endif /* BUFSIZE is large */
@end example

(Programmers often wish they could test the size of a variable or data
type in @samp{#if}, but this does not work.  The preprocessor does not
understand @code{sizeof}, or typedef names, or even the type keywords
such as @code{int}.)

@findex defined
The special operator @samp{defined} is used in @samp{#if} expressions to
test whether a certain name is defined as a macro.  Either @samp{defined
@var{name}} or @samp{defined (@var{name})} is an expression whose value
is 1 if @var{name} is defined as macro at the current point in the
program, and 0 otherwise.  For the @samp{defined} operator it makes no
difference what the definition of the macro is; all that matters is
whether there is a definition.  Thus, for example,@refill

@example
#if defined (vax) || defined (ns16000)
@end example

@noindent
would include the following code if either of the names @samp{vax} and
@samp{ns16000} is defined as a macro.  You can test the same condition
using assertions (@pxref{Assertions}), like this:

@example
#if #cpu (vax) || #cpu (ns16000)
@end example

If a macro is defined and later undefined with @samp{#undef},
subsequent use of the @samp{defined} operator returns 0, because
the name is no longer defined.  If the macro is defined again with
another @samp{#define}, @samp{defined} will recommence returning 1.

@findex #ifdef
@findex #ifndef
Conditionals that test just the definedness of one name are very common, so
there are two special short conditional commands for this case.

@table @code
@item #ifdef @var{name}
is equivalent to @samp{#if defined (@var{name})}.

@item #ifndef @var{name}
is equivalent to @samp{#if ! defined (@var{name})}.
@end table

Macro definitions can vary between compilations for several reasons.

@itemize @bullet
@item
Some macros are predefined on each kind of machine.  For example, on a
Vax, the name @samp{vax} is a predefined macro.  On other machines, it
would not be defined.

@item
Many more macros are defined by system header files.  Different
systems and machines define different macros, or give them different
values.  It is useful to test these macros with conditionals to avoid
using a system feature on a machine where it is not implemented.

@item
Macros are a common way of allowing users to customize a program for
different machines or applications.  For example, the macro
@samp{BUFSIZE} might be defined in a configuration file for your
program that is included as a header file in each source file.  You
would use @samp{BUFSIZE} in a preprocessor conditional in order to
generate different code depending on the chosen configuration.

@item
Macros can be defined or undefined with @samp{-D} and @samp{-U}
command options when you compile the program.  You can arrange to
compile the same source file into two different programs by choosing
a macro name to specify which program you want, writing conditionals
to test whether or how this macro is defined, and then controlling
the state of the macro with compiler command options.
@xref{Invocation}.
@end itemize

@ifinfo
Assertions are usually predefined, but can be defined with preprocessor
commands or command-line options.
@end ifinfo

@node Assertions
@subsection Assertions

@cindex assertions
@dfn{Assertions} are a more systematic alternative to macros in writing
conditionals to test what sort of computer or system the compiled
program will run on.  Assertions are usually predefined, but you can
define them with preprocessor commands or command-line options.

@cindex predicates
The macros traditionally used to describe the type of target are not
classified in any way according to which question they answer; they may
indicate a hardware architecture, a particular hardware model, an
operating system, a particular version of an operating system, or
specific configuration options.  These are jumbled together in a single
namespace.  In contrast, each assertion consists of a named question and
an answer.  The question is usually called the @dfn{predicate}.
An assertion looks like this:

@example
#@var{predicate} (@var{answer})
@end example

@noindent
You must use a properly formed identifier for @var{predicate}.  The
value of @var{answer} can be any sequence of words; all characters are
significant except for leading and trailing whitespace, and differences
in internal whitespace sequences are ignored.  Thus, @samp{x + y} is
different from @samp{x+y} but equivalent to @samp{x + y}.  @samp{)} is
not allowed in an answer.

@cindex testing predicates
Here is a conditional to test whether the answer @var{answer} is asserted
for the predicate @var{predicate}:

@example
#if #@var{predicate} (@var{answer})
@end example

@noindent
There may be more than one answer asserted for a given predicate.  If
you omit the answer, you can test whether @emph{any} answer is asserted
for @var{predicate}:

@example
#if #@var{predicate}
@end example

Most of the time, the assertions you test will be predefined assertions.
GNU C provides three predefined predicates: @code{system}, @code{cpu},
and @code{machine}.  @code{system} is for assertions about the type of
software, @code{cpu} describes the type of computer architecture, and
@code{machine} gives more information about the computer.  For example,
on a GNU system, the following assertions would be true:

@example
#system (gnu)
#system (mach)
#system (mach 3)
#system (mach 3.@var{subversion})
#system (hurd)
#system (hurd @var{version})
@end example

@noindent
and perhaps others.  The alternatives with
more or less version information let you ask more or less detailed
questions about the type of system software.

On a Unix system, you would find @code{#system (unix)} and perhaps one of:
@code{#system (aix)}, @code{#system (bsd)}, @code{#system (hpux)},
@code{#system (lynx)}, @code{#system (mach)}, @code{#system (posix)},
@code{#system (svr3)}, @code{#system (svr4)}, or @code{#system (xpg4)}
with possible version numbers following.

Other values for @code{system} are @code{#system (mvs)}
and @code{#system (vms)}.

@strong{Portability note:} Many Unix C compilers provide only one answer
for the @code{system} assertion: @code{#system (unix)}, if they support
assertions at all.  This is less than useful.

An assertion with a multi-word answer is completely different from several
assertions with individual single-word answers.  For example, the presence
of @code{system (mach 3.0)} does not mean that @code{system (3.0)} is true.
It also does not directly imply @code{system (mach)}, but in GNU C, that
last will normally be asserted as well.

The current list of possible assertion values for @code{cpu} is:
@code{#cpu (a29k)}, @code{#cpu (alpha)}, @code{#cpu (arm)}, @code{#cpu
(clipper)}, @code{#cpu (convex)}, @code{#cpu (elxsi)}, @code{#cpu
(tron)}, @code{#cpu (h8300)}, @code{#cpu (i370)}, @code{#cpu (i386)},
@code{#cpu (i860)}, @code{#cpu (i960)}, @code{#cpu (m68k)}, @code{#cpu
(m88k)}, @code{#cpu (mips)}, @code{#cpu (ns32k)}, @code{#cpu (hppa)},
@code{#cpu (pyr)}, @code{#cpu (ibm032)}, @code{#cpu (rs6000)},
@code{#cpu (sh)}, @code{#cpu (sparc)}, @code{#cpu (spur)}, @code{#cpu
(tahoe)}, @code{#cpu (vax)}, @code{#cpu (we32000)}.

@findex #assert
You can create assertions within a C program using @samp{#assert}, like
this:

@example
#assert @var{predicate} (@var{answer})
@end example

@noindent
(Note the absence of a @samp{#} before @var{predicate}.)

@cindex unassert
@cindex assertions, undoing
@cindex retracting assertions
@findex #unassert
Each time you do this, you assert a new true answer for @var{predicate}.
Asserting one answer does not invalidate previously asserted answers;
they all remain true.  The only way to remove an assertion is with
@samp{#unassert}.  @samp{#unassert} has the same syntax as
@samp{#assert}.  You can also remove all assertions about
@var{predicate} like this:

@example
#unassert @var{predicate}
@end example

You can also add or cancel assertions using command options
when you run @code{gcc} or @code{cpp}.  @xref{Invocation}.

@node #error Command
@subsection The @samp{#error} and @samp{#warning} Commands

@findex #error
The command @samp{#error} causes the preprocessor to report a fatal
error.  The rest of the line that follows @samp{#error} is used as the
error message.

You would use @samp{#error} inside of a conditional that detects a
combination of parameters which you know the program does not properly
support.  For example, if you know that the program will not run
properly on a Vax, you might write

@smallexample
#ifdef vax
#error Won't work on Vaxen.  See comments at get_last_object.
#endif
@end smallexample

@noindent
@xref{Nonstandard Predefined}, for why this works.

If you have several configuration parameters that must be set up by
the installation in a consistent way, you can use conditionals to detect
an inconsistency and report it with @samp{#error}.  For example,

@smallexample
#if HASH_TABLE_SIZE % 2 == 0 || HASH_TABLE_SIZE % 3 == 0 \
    || HASH_TABLE_SIZE % 5 == 0
#error HASH_TABLE_SIZE should not be divisible by a small prime
#endif
@end smallexample

@findex #warning
The command @samp{#warning} is like the command @samp{#error}, but causes
the preprocessor to issue a warning and continue preprocessing.  The rest of
the line that follows @samp{#warning} is used as the warning message.

You might use @samp{#warning} in obsolete header files, with a message
directing the user to the header file which should be used instead.

@node Combining Sources, Other Commands, Conditionals, Top
@section Combining Source Files

@cindex line control
@findex #line
One of the jobs of the C preprocessor is to inform the C compiler of where
each line of C code came from: which source file and which line number.

C code can come from multiple source files if you use @samp{#include};
both @samp{#include} and the use of conditionals and macros can cause
the line number of a line in the preprocessor output to be different
from the line's number in the original source file.  You will appreciate
the value of making both the C compiler (in error messages) and symbolic
debuggers such as GDB use the line numbers in your source file.

The C preprocessor builds on this feature by offering a command by which
you can control the feature explicitly.  This is useful when a file for
input to the C preprocessor is the output from another program such as the
@code{bison} parser generator, which operates on another file that is the
true source file.  Parts of the output from @code{bison} are generated from
scratch, other parts come from a standard parser file.  The rest are copied
nearly verbatim from the source file, but their line numbers in the
@code{bison} output are not the same as their original line numbers.
Naturally you would like compiler error messages and symbolic debuggers to
know the original source file and line number of each line in the
@code{bison} input.

@code{bison} arranges this by writing @samp{#line} commands into the output
file.  @samp{#line} is a command that specifies the original line number
and source file name for subsequent input in the current preprocessor input
file.  @samp{#line} has three variants:

@table @code
@item #line @var{linenum}
Here @var{linenum} is a decimal integer constant.  This specifies that
the line number of the following line of input, in its original source file,
was @var{linenum}.

@item #line @var{linenum} @var{filename}
Here @var{linenum} is a decimal integer constant and @var{filename}
is a string constant.  This specifies that the following line of input
came originally from source file @var{filename} and its line number there
was @var{linenum}.  Keep in mind that @var{filename} is not just a
file name; it is surrounded by doublequote characters so that it looks
like a string constant.

@item #line @var{anything else}
@var{anything else} is checked for macro calls, which are expanded.
The result should be a decimal integer constant followed optionally
by a string constant, as described above.
@end table

@samp{#line} commands alter the results of the @samp{__FILE__} and
@samp{__LINE__} predefined macros from that point on.  @xref{Standard
Predefined}.

The output of the preprocessor (which is the input for the rest of the
compiler) contains commands that look much like @samp{#line} commands.
They start with just @samp{#} instead of @samp{#line}, but this is
followed by a line number and file name as in @samp{#line}.  @xref{Output}.

@node Other Commands, Output, Combining Sources, Top
@section Miscellaneous Preprocessor Commands

@findex #pragma
@findex #ident
@cindex null command
This section describes three additional preprocessor commands.  They are
not very useful, but are mentioned for completeness.

The @dfn{null command} consists of a @samp{#} followed by a Newline, with
only whitespace (including comments) in between.  A null command is
understood as a preprocessor command but has no effect on the preprocessor
output.  The primary significance of the existence of the null command is
that an input line consisting of just a @samp{#} will produce no output,
rather than a line of output containing just a @samp{#}.  Supposedly
some old C programs contain such lines.

The ANSI standard specifies that the @samp{#pragma} command has an
arbitrary, implementation-defined effect.  In the GNU C preprocessor,
@samp{#pragma} commands are not used, except for @samp{#pragma once}
(@pxref{Once-Only}).  However, they are left in the preprocessor output,
so they are available to the compilation pass.

The @samp{#ident} command is supported for compatibility with certain
other systems.  It is followed by a line of text.  On some systems, the
text is copied into a special place in the object file; on most systems,
the text is ignored and this command has no effect.  Typically
@samp{#ident} is only used in header files supplied with those systems
where it is meaningful.

@node Output, Invocation, Other Commands, Top
@section C Preprocessor Output

@cindex output format
The output from the C preprocessor looks much like the input, except
that all preprocessor command lines have been replaced with blank lines
and all comments with spaces.  Whitespace within a line is not altered;
however, a space is inserted after the expansions of most macro calls.

Source file name and line number information is conveyed by lines of
the form

@example
# @var{linenum} @var{filename} @var{flags}
@end example

@noindent
which are inserted as needed into the middle of the input (but never
within a string or character constant).  Such a line means that the
following line originated in file @var{filename} at line @var{linenum}.

After the file name comes zero or more flags, which are @samp{1},
@samp{2} or @samp{3}.  If there are multiple flags, spaces separate
them.  Here is what the flags mean:

@table @samp
@item 1
This indicates the start of a new file.
@item 2
This indicates returning to a file (after having included another file).
@item 3
This indicates that the following text comes from a system header file,
so certain warnings should be suppressed.
@end table

@node Invocation, Concept Index, Output, Top
@section Invoking the C Preprocessor

Most often when you use the C preprocessor you will not have to invoke it
explicitly: the C compiler will do so automatically.  However, the
preprocessor is sometimes useful individually.

The C preprocessor expects two file names as arguments, @var{infile} and
@var{outfile}.  The preprocessor reads @var{infile} together with any other
files it specifies with @samp{#include}.  All the output generated by the
combined input files is written in @var{outfile}.

Either @var{infile} or @var{outfile} may be @samp{-}, which as @var{infile}
means to read from standard input and as @var{outfile} means to write to
standard output.  Also, if @var{outfile} or both file names are omitted,
the standard output and standard input are used for the omitted file names.

@cindex options
Here is a table of command options accepted by the C preprocessor.
These options can also be given when compiling a C program; they are
passed along automatically to the preprocessor when it is invoked by the
compiler.

@table @samp
@item -P
@findex -P
Inhibit generation of @samp{#}-lines with line-number information in
the output from the preprocessor (@pxref{Output}).  This might be
useful when running the preprocessor on something that is not C code
and will be sent to a program which might be confused by the
@samp{#}-lines.

@item -C
@findex -C
Do not discard comments: pass them through to the output file.
Comments appearing in arguments of a macro call will be copied to the
output before the expansion of the macro call.

@item -traditional
@findex -traditional
Try to imitate the behavior of old-fashioned C, as opposed to ANSI C.

@itemize @bullet
@item
Traditional macro expansion pays no attention to singlequote or
doublequote characters; macro argument symbols are replaced by the
argument values even when they appear within apparent string or
character constants.

@item
Traditionally, it is permissible for a macro expansion to end in the
middle of a string or character constant.  The constant continues into
the text surrounding the macro call.

@item
However, traditionally the end of the line terminates a string or
character constant, with no error.

@item
In traditional C, a comment is equivalent to no text at all.  (In ANSI
C, a comment counts as whitespace.)

@item
Traditional C does not have the concept of a ``preprocessing number''.
It considers @samp{1.0e+4} to be three tokens: @samp{1.0e}, @samp{+},
and @samp{4}.

@item
A macro is not suppressed within its own definition, in traditional C.
Thus, any macro that is used recursively inevitably causes an error.

@item
The character @samp{#} has no special meaning within a macro definition
in traditional C.

@item
In traditional C, the text at the end of a macro expansion can run
together with the text after the macro call, to produce a single token.
(This is impossible in ANSI C.)

@item
Traditionally, @samp{\} inside a macro argument suppresses the syntactic
significance of the following character.
@end itemize

@item -trigraphs
@findex -trigraphs
Process ANSI standard trigraph sequences.  These are three-character
sequences, all starting with @samp{??}, that are defined by ANSI C to
stand for single characters.  For example, @samp{??/} stands for
@samp{\}, so @samp{'??/n'} is a character constant for a newline.
Strictly speaking, the GNU C preprocessor does not support all
programs in ANSI Standard C unless @samp{-trigraphs} is used, but if
you ever notice the difference it will be with relief.

You don't want to know any more about trigraphs.

@item -pedantic
@findex -pedantic
Issue warnings required by the ANSI C standard in certain cases such
as when text other than a comment follows @samp{#else} or @samp{#endif}.

@item -pedantic-errors
@findex -pedantic-errors
Like @samp{-pedantic}, except that errors are produced rather than
warnings.

@item -Wtrigraphs
@findex -Wtrigraphs
Warn if any trigraphs are encountered (assuming they are enabled).

@item -Wcomment
@findex -Wcomment
@ignore
@c "Not worth documenting" both singular and plural forms of this
@c option, per RMS.  But also unclear which is better; hence may need to
@c switch this at some future date.  pesch@cygnus.com, 2jan92.
@itemx -Wcomments
(Both forms have the same effect).
@end ignore
Warn whenever a comment-start sequence @samp{/*} appears in a comment.

@item -Wall
@findex -Wall
Requests both @samp{-Wtrigraphs} and @samp{-Wcomment} (but not
@samp{-Wtraditional}). 

@item -Wtraditional
@findex -Wtraditional
Warn about certain constructs that behave differently in traditional and
ANSI C.

@item -I @var{directory}
@findex -I
Add the directory @var{directory} to the end of the list of
directories to be searched for header files (@pxref{Include Syntax}).
This can be used to override a system header file, substituting your
own version, since these directories are searched before the system
header file directories.  If you use more than one @samp{-I} option,
the directories are scanned in left-to-right order; the standard
system directories come after.

@item -I-
Any directories specified with @samp{-I} options before the @samp{-I-}
option are searched only for the case of @samp{#include "@var{file}"};
they are not searched for @samp{#include <@var{file}>}.

If additional directories are specified with @samp{-I} options after
the @samp{-I-}, these directories are searched for all @samp{#include}
commands.

In addition, the @samp{-I-} option inhibits the use of the current
directory as the first search directory for @samp{#include "@var{file}"}.
Therefore, the current directory is searched only if it is requested
explicitly with @samp{-I.}.  Specifying both @samp{-I-} and @samp{-I.}
allows you to control precisely which directories are searched before
the current one and which are searched after.

@item -nostdinc
@findex -nostdinc
Do not search the standard system directories for header files.
Only the directories you have specified with @samp{-I} options
(and the current directory, if appropriate) are searched.

@item -nostdinc++
@findex -nostdinc++
Do not search for header files in the C++-specific standard directories,
but do still search the other standard directories.
(This option is used when building libg++.)

@item -D @var{name}
@findex -D
Predefine @var{name} as a macro, with definition @samp{1}.

@item -D @var{name}=@var{definition}
Predefine @var{name} as a macro, with definition @var{definition}.
There are no restrictions on the contents of @var{definition}, but if
you are invoking the preprocessor from a shell or shell-like program you
may need to use the shell's quoting syntax to protect characters such as
spaces that have a meaning in the shell syntax.  If you use more than
one @samp{-D} for the same @var{name}, the rightmost definition takes
effect.

@item -U @var{name}
@findex -U
Do not predefine @var{name}.  If both @samp{-U} and @samp{-D} are
specified for one name, the @samp{-U} beats the @samp{-D} and the name
is not predefined.

@item -undef
@findex -undef
Do not predefine any nonstandard macros.

@item -A @var{predicate}(@var{answer})
@findex -A
Make an assertion with the predicate @var{predicate} and answer
@var{answer}.  @xref{Assertions}.

@noindent
You can use @samp{-A-} to disable all predefined assertions; it also
undefines all predefined macros that identify the type of target system.

@item -dM
@findex -dM
Instead of outputting the result of preprocessing, output a list of
@samp{#define} commands for all the macros defined during the
execution of the preprocessor, including predefined macros.  This gives
you a way of finding out what is predefined in your version of the
preprocessor; assuming you have no file @samp{foo.h}, the command

@example
touch foo.h; cpp -dM foo.h
@end example

@noindent 
will show the values of any predefined macros.

@item -dD
@findex -dD
Like @samp{-dM} except in two respects: it does @emph{not} include the
predefined macros, and it outputs @emph{both} the @samp{#define}
commands and the result of preprocessing.  Both kinds of output go to
the standard output file.

@item -M
@findex -M
Instead of outputting the result of preprocessing, output a rule
suitable for @code{make} describing the dependencies of the main
source file.  The preprocessor outputs one @code{make} rule containing
the object file name for that source file, a colon, and the names of
all the included files.  If there are many included files then the
rule is split into several lines using @samp{\}-newline.

This feature is used in automatic updating of makefiles.

@item -MM
@findex -MM
Like @samp{-M} but mention only the files included with @samp{#include
"@var{file}"}.  System header files included with @samp{#include
<@var{file}>} are omitted.

@item -MD
@findex -MD
Like @samp{-M} but the dependency information is written to files with
names made by replacing @samp{.c} with @samp{.d} at the end of the
input file names.  This is in addition to compiling the file as
specified---@samp{-MD} does not inhibit ordinary compilation the way
@samp{-M} does.

In Mach, you can use the utility @code{md} to merge the @samp{.d} files
into a single dependency file suitable for using with the @samp{make}
command.

@item -MMD
@findex -MMD
Like @samp{-MD} except mention only user header files, not system
header files.

@item -H
@findex -H
Print the name of each header file used, in addition to other normal
activities.

@item -imacros @var{file}
@findex -imacros
Process @var{file} as input, discarding the resulting output, before
processing the regular input file.  Because the output generated from
@var{file} is discarded, the only effect of @samp{-imacros @var{file}}
is to make the macros defined in @var{file} available for use in the
main input.

@item -include @var{file}
@findex -include
Process @var{file} as input, and include all the resulting output,
before processing the regular input file.  

@item -idirafter @var{dir}
@findex -idirafter
@cindex second include path
Add the directory @var{dir} to the second include path.  The directories
on the second include path are searched when a header file is not found
in any of the directories in the main include path (the one that
@samp{-I} adds to).

@item -iprefix @var{prefix}
@findex -iprefix
Specify @var{prefix} as the prefix for subsequent @samp{-iwithprefix}
options.

@item -iwithprefix @var{dir}
@findex -iwithprefix
Add a directory to the second include path.  The directory's name is
made by concatenating @var{prefix} and @var{dir}, where @var{prefix}
was specified previously with @samp{-iprefix}.

@item -lang-c
@itemx -lang-c++
@itemx -lang-objc
@itemx -lang-objc++
@findex -lang-c
@findex -lang-c++
@findex -lang-objc
@findex -lang-objc++
@findex #import
Specify the source language.  @samp{-lang-c++} makes the preprocessor
handle C++ comment syntax (comments may begin with @samp{//}, in which
case they end at end of line), and includes extra default include
directories for C++; and @samp{-lang-objc} enables the Objective C
@samp{#import} command.  @samp{-lang-c} explicitly turns off both of
these extensions, and @samp{-lang-objc++} enables both.

These options are generated by the compiler driver @code{gcc}, but not
passed from the @samp{gcc} command line.

@item -lint
Look for commands to the program checker @code{lint} embedded in
comments, and emit them preceded by @samp{#pragma lint}.  For example,
the comment @samp{/* NOTREACHED */} becomes @samp{#pragma lint
NOTREACHED}.

This option is available only when you call @code{cpp} directly;
@code{gcc} will not pass it from its command line.

@item -$
@findex -$
Forbid the use of @samp{$} in identifiers.  This is required for ANSI
conformance.  @code{gcc} automatically supplies this option to the
preprocessor if you specify @samp{-ansi}, but @code{gcc} doesn't
recognize the @samp{-$} option itself---to use it without the other
effects of @samp{-ansi}, you must call the preprocessor directly.

@end table

@node Concept Index, Index, Invocation, Top
@unnumbered Concept Index
@printindex cp

@node Index,, Concept Index, Top
@unnumbered Index of Commands, Macros and Options
@printindex fn

@contents
@bye
